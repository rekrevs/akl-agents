\documentstyle[a4,11pt]{report}

\font\ssi=cmssi10 at 11pt

\def\inp{\ssi}

\begingroup
\catcode`\|=13
\catcode`\_=13
\catcode`\<=13
\catcode`\>=13

\gdef\prog{\sf%
\catcode`\|=13%
\def|{{\tt \char`|}}%
\catcode`\_=13%
\def_{{\tt \char`_}}%
\catcode`\<=13%
\def<{{\tt \char`\<}}%
\catcode`\>=13%
\def>{{\tt \char`\>}}%
}
\endgroup

\def\sub#1{$#1$}

\newenvironment{program}{%
\begingroup%
\prog%
\def\nl{\\[\medskipamount]}%
\begin{tabbing}%
\hskip1cm\=\hskip1cm\=\hskip1cm\=\hskip1cm\=\kill}}}
\def\AGGREGATE{\mbox{\bf aggregate}}
\def\BAGOF{\mbox{\bf bagof}}
\def\nt#1{\langle\mbox{\em #1\/}\rangle}
\def\bnf{\mathrel{::=}}
\def\logand{\mathrel{\&}}
\def\implies{\supset}

\def\instruction#1{\medskip\noindent{\bf #1}}

\title{Design of a\\
Sequential Prototype Implementation\\
of the Andorra Kernel Language\\
(DRAFT)}

\author{Bj\"orn Danielsson, Sverker Janson, Johan Montelius, Seif Haridi\vspace{0.3cm}\\
Swedish Institute of Computer Science\\
Box 1263, S-164 28 KISTA, Sweden\\
E-mail: \{bd, sverker, jm, seif\}@sics.se}

\begin{document}

\maketitle

\tableofcontents

\chapter{Introduction} \label{ch:intro}

The Andorra Kernel Language (AKL) is a general combination of
search-orien\-ted nondeterministic languages, such as Prolog, and
process-oriented com\-mit\-ted-choice languages, such as GHC.  For an
introduction to AKL from this perspective, see \cite{jaha91PEPMA}.
For an extensive formal treatment, see \cite{fra91PEPMA}.  A brief
formal language definition is provided in appendix \ref{app:akldef}.

This paper describes a sequential prototype implementation of AKL.  It consists
of a compiler from AKL to an abstract machine, an emulator for this
abstract machine written in C, a run-time system for the emulator, and
a library written in AKL.

\section{Design Space}

The design space for programming language implementation is vast.
The following sections contain a few highlights.

\subsection{Interpreter or Compiler}

Although writing an interpreter usually means less implementation
work, the execution overhead of an interpreter can hide severe
inefficiencies of a language and its implementation.  As the purpose
was to evaluate the efficiency of AKL, we decided that the
implementation should be based on a compiler.  The next question is
whether the target should be a real machine, or an emulator of an
abstract machine.  The first alternative is of course the ``real''
solution, but is not portable.  The second alternative is less
efficient than the first, but still an order of magnitude faster than
an interpreter.  Development time is also much shorter.

We chose to compile AKL to a high level abstract machine similar in
spirit to the Warren Abstract Machine (WAM) \cite{wam}.  The emulator
of the abstract machine is written in C.  The cost of this approach is
well understood.  It allows inefficiencies to be spotted and
optimizations to be designed in the framework of the abstract machine.
By keeping the abstract machine close to WAM, we have been able to
make good use of a simple existing compiler from Prolog to WAM.  This
compiler was adapted for the Prototype Abstract Machine (PAM) with
minor changes only, and is still written in Prolog.

PAM has no optimizations for common cases, such as flat guards.  By
providing specialized instructions for flat guards, an optimization
can be achieved which is similar to a shallow backtracking
optimization in Prolog \cite{shallow}, but of greater importance in
AKL.  It has been omitted from this design, as being a complication of
importance in principal, but will be discussed.

\subsection{Memory Management}

Although the use of many data objects would allow stack-based memory
allocation in the implementation, this would restrict the
implementation, and in particular it would complicate future
experiments with the execution order.

We decided that heap-based memory allocation would be sufficient for
our purposes.  As it turns out, garbage collection does not impair the
efficiency of the implementation, and there has been little need to
optimize it.  In an improved implementation, and when the requirements
of real applications are considered, garbage collection is likely to
need renewed attention, and when the issue of execution order is
completely investigated, it will be desirable to employ stack-based
memory allocation whenever possible.

\subsection{Nondeterminism}

AKL provides don't know nondeterminism through the non-determinate
promotion rule.  The two obvious alternatives for implementing this
mechanism are either to use copying as suggested by the formulation of
the rule, or to use some sharing scheme, reminiscent of trailing and
backtracking in Prolog.

We decided to use copying, for simplicity.  However, because copying
can change the computational complexity of a program, we have also
designed schemes for sharing.  An ideal implementation should probably
use a sharing scheme.


\chapter{A Sequential Execution Model} \label{ch:exemodel}

Write down an initial configuration. Make transitions with respect to
the leftmost possible subgoal. Stop when no transition is possible.
This is a perfectly viable {\em execution model}, which respects the
computation model but makes specific decisions about which step to
make when.

Thus, by an execution model we mean a refinement of the computation model, 
where more specific control principles are used.  By an execution we mean a 
computation following these control principles.

The execution model presented here tries to avoid frequent context
switches.  It goes as follows.  Use two pens, one red (``active'') and
one black (``suspended'').  Use red for the initial configuration, for
the goals created by local forking, and for the atoms promoted by
determinate promotion.  The environment of a goal only takes black
constraint atoms into account.  Keep your finger pointing at the goal
which was most recently introduced by a computation rule, starting by
pointing to the outermost and-box in the initial configuration.  The
box which is pointed at is called the current box.

Repeat the following steps.
%
\begin{enumerate}
\item
Take the leftmost transition with respect to a red goal in the
current box, moving your finger as necessary.
\item
If no transition is possible with respect to a red goal, make
any transition with respect to a black goal in the current box, moving
your finger as necessary.
\item
Otherwise, rewrite the current box in black, and move your
finger to the parent box.
\end{enumerate}
%
By making a distinction between active and suspended goals, and
postponing actions on suspended goals, context switches are avoided.
The active goals are processed in an unbroken left to right manner,
allowing efficient implementation.

The following sections formalise the execution model, introducing the
notions of a worker and tasks and their execution.

The philosophy of the execution model is that each operation should,
intuitively, have the same algorithmic complexity as a corresponding
operation in an abstract machine.  Therefore auxiliary control
information is added, which is not strictly necessary to express the
control desired.  The reasons for adding control information will be
explained.

\section{Goals}

Goals are given {\em labels}, a label being anything which may serve
as an identifier.  Labels allow us to identify, informally, goals in
different, subsequent configurations, in a manner not unlike the way
you, dear reader, identify yourself with the person who started
reading this chapter a minute ago.  A suitable mental model is to
think of a label as the address of a computer representation of the
goal.

Conceptually, upon creation, each goal is given a label which is
unique for the current computation.  Rewriting the interior of a goal
does not change its label.  How rules preserve labels should be quite
intuitive except maybe for the choice splitting rule which copies
goals. (The left copy is new, and is given new labels.)

Goals occurring in a computation state are {\em live} (in this
computation state).  Other goals are {\em dead}.  When a goal is newly
created, it is {\em untried}; later it may be regarded as {\em tried}.

\section{Tasks}

A {\em task} is a pair of a procedure and a goal.  The purpose of the
procedure is to evaluate the applicability of some computation rules
to the goal, and if they are applicable, also apply them.  The
procedure may also create new tasks.  By {\em processing} a task we
mean executing the procedure with respect to the goal.

Variables have associated with them a list of tasks that should be
considered when a variable is constrained.  Associating a task with a
variable is called {\em suspending} it on the variable.  The associated
tasks are called {\em suspensions}.

\section{Workers}

A {\em worker} is an automaton that processes tasks.

A worker is located in an and-box or in a choice-box (the {\em current
box}).  The current box is changed by {\em moving} between boxes.  When
moving to a box which is contained within the current box, the worker
is said to move {\em down}.  When moving to a box containing the
current box, the worker is said to move {\em up}.  Otherwise, the
worker moves up to the closest and-box containing the destination box,
and then down.  When moving, the worker moves step by step, to each
intermediate box.

A worker has associated with it a {\em constraint window}, which is a
representation of the constraint environment of the goals in the
current box.  When a worker moves down to an and-box, the constraints
it contains are {\em installed} in the window.  When a worker moves up
from an and-box, the constraints it contains are {\em deinstalled}.
When a worker promotes the current and-box, the constraints it
contains are {\em promoted}.  The nature of these operations
(installation, deinstallation, and promotion) depend on the constraint
system.

A worker associates {\em tasks} with and-boxes and with choice-boxes.
Tasks are said to be {\em in} boxes (from the point of view of a
worker).  A worker keeps track of its tasks, and which tasks are in
which boxes.  When a task has been processed, or when it is in a box
which is deleted because of failure, it is deleted.  In this execution
model, tasks are processed in a {\em last-in first-out} fashion.  The
worker processes all the tasks in a box before moving to a parent box.
This allows the worker to keep its tasks in a stack.

The first task is always to try the and-box which is the initial
configuration.  This procedure is defined in the next section.
The worker continues processesing tasks until the task stack
is empty.

\section{Task Procedures} \label{sec:taskproc}

In this section, the {\em task procedures} are described.  We also
describe auxiliary procedures for establishing stability of and-boxes,
and for finding a candidate for nondeterminate promotion.

The main kind of task procedures are called {\em trying an X}, where
{\em X} is a goal.  There is one procedure for each kind of goal.
When the worker processes a task of this kind, it implicitly moves to
{\em X} if {\em X} is an and-box or a choice-box, and otherwise to the
closest surrounding and-box or choice-box.

The procedures are described in informal pseudo-code.  Blocks,
conditionals, and repetition statements, etc., have C syntax, but
natural language is used for most statements and expressions.  We will
make use of anaphora and implicit references to boxes to keep the
language informal.

For brevity, when a task is created which will be the next to be
processed, we simply say that it is tried.  Other tasks that are
created are implicitly entered in the current box.

\paragraph{Trying an And-Box} This procedure finds all non-suspended work
in an and-box, and creates appropriate tasks.
%
\begin{program}
try an andbox \{\\
\>if (the and-box contains untried atomic goals)\\
\>\>try the first;\\
\>else if (there are tasks in the and-box)\\
\>\>continue with them;\\
\>else if (the and-box is solved)\\
\>\>try the parent guarded goal;\\
\>else if (the and-box is known to be stable)\\
\>\>attempt choice splitting;\\
\}
\end{program}

\paragraph{Trying a Program Atom}
\begin{program}
try a program atom \{\\
\>apply the local forking rule to the atom;\\
\>move to the new choice-box;\\
\>create tasks for trying each of the and-boxes\\
\>\>in the new guarded goals, from the last to the first\\
\}
\end{program}

\paragraph{Trying a Constraint Atom} 
\begin{program}
try a constraint atom \{\\
\>install the constraint atom;\\
\>if (installation fails)\\
\>\>propagate failure;\\
\>else \{\\
\>\>examine suspensions on variables that were constrained;\\
\>\>if (external variables were constrained)\\
\>\>\>suspend the current and-box on these variables;\\
\>\>try the current and-box;\\
\>\}\\
\}
\end{program}

\paragraph{Trying a Wait Guarded Goal}
\begin{program}
try a wait guarded goal \{\\
\>if (the parent choice-box is determinate)\\
\>\>promote the guard;\\
\}
\end{program}

\paragraph{Trying a Cut Guarded Goal}
\begin{program}
try a cut guarded goal \{\\
\>if (the guard is quiet)\\
\>\>apply the cut rule;\\
\>\>remove all tasks from the parent choice-box;\\
\>\>if (the parent choice-box is determinate)\\
\>\>\>promote the guard;\\
\>\}\\
\}
\end{program}

\paragraph{Trying a Commit Guarded Goal}
\begin{program}
try a commit guarded goal \{\\
\>if (the guard is quiet)\\
\>\>apply the commit rule;\\
\>\>promote the guard;\\
\>\}\\
\}
\end{program}

\paragraph{Attempting Choice Splitting} An attempt is made to
apply choice splitting within a stable and-box, and create appropriate
tasks.  If this is not possible, its parent choice-box is tried.  (The
references to the boxes involved are maybe a little hard to follow.)
%
\begin{program}
attempt choice splitting \{\\
\>try to find a candidate for choice splitting in the and-box;\\
\>if (there is no candidate)\\
\>\>try the parent choice-box;\\
\>else \{\\
\>\>call the candidate A, its parent choice-box B,\\
\>\>\>B's parent and-box C, and C's parent choice-box D;\\
\>\>apply the choice splitting rule wrt A;\\
\>\>call the copy of A, A';\\
\>\>move to choice-box D;\\
\>\>if (choice-box B is determinate)\\
\>\>\>create a task for promotion of the remaining branch;\\
\>\>else if (and-box C is stable)\\
\>\>\>create a task for attempting choice splitting\\
\>\>\>\>within C;\\
\>\>promote the and-box A';\\
\>\}\\
\}
\end{program}

\paragraph{Examine Supensions on a Variable}
\begin{program}
examine suspensions on a variable \{\\
\>for (each of the suspensions on the variable)\\
\>\>if (the associate goal is in the current box)\\
\>\>\>move the task to the current box;\\
\}
\end{program}

\paragraph{Propagating Failure}
On the entry to this procedure, either the condition for environment
synchronisation or the condition for failure propagation is met.  Both
rewrite an and-box to $\FAIL$.
\begin{program}
propagate failure \{\\
\>apply either environment syncronisation or failure propagation;\\
\>the deleted and-box is now dead;\\
\>move to the parent choice-box;\\
\>apply the choice elimination rule;\\
\>if (there is only one guarded goal in the choice-box)\\
\>\>try this guarded goal;\\
\>else if (there are no guarded goals in the choice-box)\\
\>\>propagate failure;\\
\}
\end{program}

\paragraph{Promotion}
\begin{program}
promote a solved and-box \{\\
\>move to the parent and-box of the solved and-box;\\
\>apply the determinate promotion rule;\\
\>if (\=there are promoted constraints\\
\>    \>for variables that are external to the solved and-box)\\
\hskip1cm\=\hskip1cm\=\hskip1cm\=\hskip1cm\kill
\>\>move their associated tasks that refer to subgoals\\
\>\>\>of the current box to the current box;\\
\>try the current box;\\
\}
\end{program}

\paragraph{Waking Supensions on a Variable}
\begin{program}
wake suspensions on a variable \{\\
\>for (each of the tasks suspended on the variable)\\
\>\>if (the associate goal is alive)\\
\>\>\>if (its associated goal is in the current box)\\
\>\>\>\>move the task to the current box;\\
\}
\end{program}

\paragraph{Establishing Stability}
There are several possible schemes for establishing stability that
vary in precision.  With each and-box is associated information about
suspensions on external variables occurring within it.  When creating
a suspension on an external variable, this has to be recorded in all
ancestor and-boxes for which the variable is external.  The schemes
vary in what degree they take care of the waking of a suspension and
the failure of an and-box.  When a suspension is woken, this should be
recorded in ancestor and-boxes.  When an and-box fails, it should be
recorded in all ancestor and-boxes that all suspensions on external
variables in the and-box have disappeared.  If either of these last
two cases is not taken care of, boxes will be deemed to be unstable
where they are not---it can not be established that they are stable.

\paragraph{Finding a Candidate}
By a {\em candidate} is meant a solved and-box that may be promoted by
the nondeterminate promotion rule (or here singled out by the choice
splitting rule).  In the execution model, the left-most candidate
should be chosen.  This involves examining the ``leaves'' in the
stable and-box left-to-right, choosing the first candidate.

\section{Some Properties of the Execution Model}

Untried atomic goals in an and-box are always a {\em contiguous
sub-sequence} of the local goals, which is accessed from the left to
the right.  This can be realised by reasoning as follows.  Initially,
all goals in an and-box are untried atomic goals.  A body (consisting
of untried atomic goals) will only be promoted when trying the
left-most atomic goal, and will then replace this goal.  Suspended
goals within the and-box are only retried when all untried atomic
goals have been tried.

In the implementation, this property makes it possible to have a
single-linked list representation for untried atomic goals.  The
position of this list within the other local goals can be kept in a
register.

Only constraints on external variables need to be explicit.
Constraints on local variables are only referred to quite implicitly.
For example, instead of creating a binding $X = t$ of a local variable
$X$, the term $t$ can be {\em substituted} for all occurrences of the
variable $X$.  Also note that the position of a tried binding among
the local goals is quite irrelevant.

In the implementation, substituting a value for a local variable only
involves replacing the value slot in the object representing the
variable with the value.  Also, the constraints (for external
variables) can be kept in a separate list.  The test for quietness is
to see if this list is empty.  The test for being solved is to see if
the list of other goals is empty.

With the adapted computation model, there are {\em no or-boxes}, and
there is always exacly one and-box per guarded goal.

In the implementation, we may collapse the notions of guarded goal and
and-box, and let them be represented by a single object (which, by
abuse of terminology, is called an and-box).


\chapter{A Prototype Abstract Machine} \label{ch:absmachine}

The abstract machine of the prototype implementation, the Prototype
Abstract Machine (PAM), is an implementation of the execution model.
The abstract machine describes requirements on the representations of
programs, of the configuration and of the binding window. The abstract
machine also describes how a worker executes tasks given these
requirements.

The abstract machine is deliberately not detailed.
The description does not specify how constituents are stored nor how
they are explicitly represented. The abstract machine only specifies the 
requirements on the constituents; the exact representation is up to 
the implementation. 

By making a distinction between an abstract machine and its
implementation, the abstract machine is not impaired by
implementation-dependent details. We can concentrate on its
functionality.

The design decisions taken for the abstract machine are:
%
\begin{itemize}
\item Only support the execution of a single worker. This simplifies the
representation of goals in and-boxes and the representation of variable
bindings.
\item Handle equality in the domain of rational trees explicitly. 
Specialised instructions improve efficiency as usual.
\end{itemize}
%
The first section describes the properties of the configuration but
not how or why constituents are created. The second section describes
the logical properties of the binding window. This includes the
constraint system, how guarded goals are suspended and how the stability 
property is maintained.  The third section describes how choice
splitting is handled. The fourth section describes the execution state
and the steps taken by a worker. The last section describes the
instructions of the abstract machine.


\section{The Configuration}

In the abstract machine we collapse the notions of a guarded goal and
the and-box it has in its guard, and represent them using a single
object, which will, by abuse of terminology, be called an {\em
and-box}.

The configuration consists of a tree of choice-boxes and and-boxes.
Each box can reach its children and its parent.  The root of the tree
is a choice-box called {\em root}.

In a choice-box the tried guarded goals are represented by a sequence
of and-boxes. The untried guarded goals are represented implicitly by
a sequence of instructions and the arguments of the program atom. In
an and-box the untried program atoms and constraint atoms are
represented implicitly by a sequence of abstract machine instructions
and a tuple of permanent registers.

\subsection*{Choice-Boxes}

A choice-box contains the representation of tried guarded goals (i.e.,
and-boxes), the information needed to try the remaining guarded goals
and a reference to its parent and-box.

\begin{center} \begin{tabular}{|l|l|} \hline
Choice-box \\
\hline
\hline
and-boxes \\
choice continuation \\
parent and-box\\
\hline
\end{tabular} \end{center}

Untried guarded goals are represented implicitly by a {\em choice
continuation}. When the choice continuation is consumed, we refer to
it as a {\em null continuation}.

\begin{center} \begin{tabular}{|l|l|} \hline
Choice-continuation\\
\hline
\hline
instructions of remaining guarded goals\\
arguments of the guarded goal \\
\hline
\end{tabular} \end{center}

The root choice-box needs only the list of tried guarded goals. It has
no parent and-box and a null continuation.

\subsection*{And-Boxes}

An and-box contains the state of the guard computation, stability
information for the and-box, constraints, a sequence of choice-boxes,
a sequence of and-continuations and a reference to its parent
choice-box.

\medskip
\begin{center} \begin{tabular}{|l|l|} \hline
And-box \\
\hline
\hline
state\\
stability\\
constraints \\
choice-boxes \\
and-continuations \\
parent choice-box \\
\hline
\end{tabular} \end{center}

An and-box is either {\em alive} or {\em dead}. This is called the
{\em state} of the and-box.  An and-box is initially alive and can
become dead.

The {\em stability information} maintained in an and-box
is a (possibly approximate) account of the constraints on variables
external to the and-box occurring in the subtree it forms.  How to
represent and maintain this information is described in section
\ref{absstab}.

The {\em constraints} in an and-box represent constraints on external
variables. (Constraints on local variables are only recorded on the
variables.)

The {\em choice-boxes} in an and-box are ordered. The representation
of this sequence should allow left-to-right traversal.  This is
necessary in order to find the leftmost candidate for choice
splitting.

The sequence of {\em and-continuations} represents a sequence of
untried program atoms and constraint atoms.
The last continuation includes the representation of
the guard operator and body of the guarded goal. 

The sequence of and-continuations has an implicit position in the
sequence of choice-boxes.  This position is called the {\em insertion
point}. The insertion point is maintained by the worker, and is not
represented in the and-box.

\medskip
\begin{center} \begin{tabular}{|l|l|} \hline
And-continuation \\
\hline
\hline
instructions\\
permanent registers \\
\hline
\end{tabular} \end{center}

The untried atomic goals are represented by a sequence of instructions
and a tuple of {\em permanent registers} with respect to which the
instructions are to be executed.


\subsection*{Constraints}

In this prototype abstract machine, constraints are restricted to
equality on rational trees.  A constraint structure, representing an
equality, contains a variable and a term.

\medskip
\begin{center} \begin{tabular}{|l|l|} \hline
Constraint \\
\hline
\hline
variable \\
term \\
\hline
\end{tabular} \end{center}


\subsection*{Terms}

There are three types of terms: atomic, structures and
variables.  Implicitly, all terms have an identity.  This is the only
property of an atomic term. A structure has two additional properties:
a functor and an argument tuple. A variable has three additional
properties: the current value, a set of suspensions and an environment
identifier.

\begin{center} \begin{tabular}{|l|l|} \hline
Variable\\
\hline
\hline
current value\\
suspensions\\
environment\\
\hline
\end{tabular} \end{center}

A variable is either {\em unbound} or {\em bound}.  If the set of
suspensions is empty the variable is {\em unconstrained}; if the set
is not empty the variable is {\em constrained}.

The environment identifier of a variable identifies the and-box in
which the variable is local.  A variable is local to the and-box in
which it was created; if the and-box is promoted the variable becomes
local to the parent and-box.

\begin{center} \begin{tabular}{|l|}\hline
Suspension\\
\hline
\hline
and-box\\
\hline
\end{tabular} \end{center}

Each suspension refers to a constrained and-box which contains
constraints on this variable.

\subsection*{Definitions}

A definition is represented by a sequence of instructions.
Subsequences are used to represent untried guarded goals and untried
atomic goals. The instructions of the abstract machine can be divided
into four groups: decision, procedural, guard and term instructions.

The decision instructions are used to select a sequence of guarded
goals. The instructions also set up the configuration to execute
the instructions of the guarded goal. This means building the
choice-box of the goal and the and-boxes of the guarded goals.

The term instructions represent constraint atoms, and perform the
related actions.  The procedural instructions pass control to a
definition or the next and-continuation.  The guard instructions
determine whether or not the and-box is to be promoted and/or if
sibling and-boxes are to be pruned.

A complete description of the instructions is found in section
\ref{absinstr}.

\section{Binding window}

The notion of a {\em binding window} is a restriction of a constraint
window to bindings, i.e., equalities relating variables to terms.

Execution is restricted to a single and-box or choice-box at a time.
The and-box or choice-box in which execution takes place is called the
{\em current and-box} or the {\em current choice-box}. The branch from
the root to the current box is called the {\em current branch}. 

The binding window is a worker's view of the constraints of the current
branch. This view changes as the worker moves around in the
configuration. The most complex aspect of the abstract machine is how to
represent and maintain the binding window of a worker.

\subsection*{Constraint system}

To start with we must describe the constraint system.  The abstract
machine only describes how equality in the domain of rational
trees is handled. This is a simple yet powerful constraint
system. Other constraint systems are easily added in a uniform manner.

When a constraint atom is tried the two terms of the equality are {\em
unified}. A variable is unified with a term by making the term the
current value of the variable, the variable is {\em bound} to the
term.  By {\em dereferencing} a variable we mean following any
variable bindings until a non-variable or an unbound is reached.

A unification of two terms is always done with respect to the current
and-box. The description of unification is therefore in terms of local
and external variables. To determine whether a variable is local the
environment of the variable must be examined. If the environment
identifier refers to the current and-box, the variable is a {\em local
variable}, otherwise it is an {\em external variable}. The requirement
on a unification is simple.

\begin{itemize}
\item[U] A unification must never bind an external variable 
to a local variable.
\end{itemize}



\subsection*{Moving around}

To maintain a consistent binding window as the worker moves around in
the configuration the following procedures are performed.


\begin{itemize}

\item[M1] An and-box is installed by examining the constraints of the and-box. 
If the variable is unbound it is bound to the term. If the variable is
bound the constraint is removed and the two terms are unified. If the
unification fails the remaining uninstalled constraints are removed and the
installation terminates with failure

\item[M2] If an external variable is bound, a constraint is added 
to the and-box. The constraint contains the variable and the term to
which it was bound.

\item[M3] An and-box is de-installed by unbinding the variables of the 
constraints.

\item[M4] When an and-box is promoted the constraints containing a variable 
that is external to the parent and-box are moved to the parent and-box.

\item[M5] If an and-box fails the variables of the constraints are unbound.

\end{itemize}

The procedures described allow a worker to maintain a consistent
binding window. It does not however solve the problem of {\em suspending}
an and-box on a variable or waking an and-box when a variable becomes bound. 

\subsection*{Waking}

To {\em wake} a {\em suspended} and-box a {\em wake stack} is used. The
wake stack is divided into segments. Each segment belongs to an and-box
in the current branch. A wake entry consists of a reference to a suspended 
and-box.

\begin{center} \begin{tabular}{|l|}\hline
Wake entry \\
\hline
\hline
and-box\\
\hline
\end{tabular} \end{center}

Before describing the procedures for the wake stack we need a
definition of {\em scope}. The current and-box is in the scope of
itself. An and-box is said to be in the scope of the current and-box
if the and-box is alive and the parent of the parent choice-box is in
the scope of the current and-box.

To detect if an and-box is in the scope of the current and-box the
path from the and-box to the root is examined. If there is a path of
living and-boxes from the and-box to the current and-box the and-box is 
in the scope of the current and-box.


\begin{itemize}

\item[W1] If an external unconstrained variable is bound or if an
external variable is bound to an external unconstrained variable a
constraint is added to the current and-box and a suspension is added
to the variable.

\item[W2] If an external constrained variable is bound or if an external 
variable is bound to an external constrained variable the suspensions
referring to and-boxes in the scope of the current and-box are
removed. For each removed suspension a wake entry is created for each
of the suspended and-box.  A constraint is added to the current
and-box and a suspension is added to the variable.

\item[W3] If a local constrained variable is bound 
all suspensions are removed and a wake entry is
created for each of the suspended and-boxes. The state of the variable is
changed to unconstrained.

\end{itemize}

To handle suspensions correctly the same considerations must be observed
when an and-box is promoted. When an and-box is promoted we will say
that the bindings recorded in the constraints are promoted.

\begin{itemize}

\item[W4] If a binding of an external constrained variable is promoted,
the suspensions referring to and-boxes in the scope of the current
and-box are removed. For each removed suspension a wake entry is
created for the suspended and-box. A constraint is added to the
current and-box and a suspension is added to the variable.

\item[W5] If a binding of a local constrained variable is promoted 
all suspensions are removed and a wake entry is created for each of the
suspended and-boxes. 

\end{itemize}

All wake entries of an and-box must be handled before the worker is
allowed to move up to the parent choice-box.


\subsection*{Stability} \label{absstab}

A naive way of detecting stability of the current and-box is to
traverse all and-boxes in the scope of the current and-box. If a
constraint is found for a variable that is external to the current
and-box the current and-box is unstable. The overhead of this
procedure is in the general case unacceptable. Instead of doing a
traversal when stability should be detected one may choose to incur the
inevitable penalty when variables are constrained.

The stability information in an and-box can contain a set of {\em
marks}. Each mark represents a constraint in the scope of (but not
including) the and-box. All marks of a constraint are connected and 
associated to the constraint.  An and-box is stable if (and
only if) the set of constraints is empty and the set is marks is
empty. The set of marks can be maintained as follows.


\begin{itemize}

\item[S1] When a constraint is added to an and-box a mark is added to 
all and-boxes in the path from (but not including) the and-box to (but not
including) the and-box for which the variable is local. The set of marks
is associated to the constraint.

\item[S2] If a constraint is removed from an and-box the marks of the 
constraint are removed in all and-boxes.

\item[S3] If an and-box fails the constraints of the and-box and the 
marks of the and-box are examined. The marks of all constraints are
removed in all and-boxes. The marks connected to the marks of the
and-box are removed in all and-boxes above the and-box.

\end{itemize}

An and-box thus maintains a record of all constraints on external
variables in any and-box that is in the scope of the and-box. A
penalty is only incured when a constraint is added or removed. Observe that
marks are created only if a constraint is added for a variable that is
external not only to the current and-box but to the parent of the
current and-box.

This scheme can be modified if one relaxes the requirement that
stability always should be detected. If an and-box only records if a
constraint has been added, not which constraint was added, the
stability information is reduced to a counter.


\begin{itemize}

\item[S'1] When a constraint is added to an and-box the counter is 
incremented in all and-boxes in the path from (but not including) the
and-box to (but not including) the and-box for which the variable is
local.

\item[S'2] If a constraint is removed from an and-box the counters 
are decremented in all and-boxes in the path from (but not including) the
and-box to (but not including) the and-box for which the constrained
variable is local.

\item[S'3] If an and-box fails the counters are decremented for each 
constraint of the and-box in all and-boxes in the path from (but not
including) the failing and-box to (but not including) the and-box for
which the constrained variable is local.

\end{itemize}

This scheme is not complete. If an and-box fails the counters are only
decremented for each of the constraints of the failing and-box. The
constraints of and-boxes in the scope of the failing and-box are not
considered. The counter scheme is however complete if no constraint is
added for a variable that is external not only to the current and-box
but also to the parent of the current and-box.

One can simplify the scheme even further. The counter can be reduced
to a flag or a dirty bit.

\begin{itemize}

\item[S''1] When a constraint is added to an and-box all and-boxes in the 
path from (but not including) the and-box to (but not including) the and-box
for which the variable is local are marked as unstable.

\end{itemize}

An and-box that has been marked as unstable will never be marked as
stable again.  An and-box marked as unstable is thus not necessarily unstable.
The information of stability is lost once an and-box has become unstable. 


\subsection*{Adding a trail}

One can observe that the binding of an external variable is an
expensive operation. The constraint added to the and-box and the
suspension added to the variable are only needed if the and-box is
suspended and later installed. The stability of parent and-boxes must
only be updated when an and-box suspends. To improve efficiency in the
common case where an and-box fails or is promoted without being
suspended a {\em trail stack} is used.

\begin{center} \begin{tabular}{|l|}\hline
Trail entry \\
\hline
\hline
variable \\
\hline
\end{tabular} \end{center}

The trail stack is a sequence of sets. Each set belongs to an and-box
of the current branch. Each trail entry contains a reference to a
variable. The trail stack is used as follows.


\begin{itemize}

\item[T1] If an external variable is bound no constraint is added 
to the and-box. Instead the variable is recorded on the trail.
Wake entries are added if needed.

\item[T2] If an and-box is suspended a new constraint is added for 
each trailed variable. Suspensions are added to the variables and
the  stability information is updated.

\item[T3] When an and-box is promoted the trail entries containing a variable 
that is external to the parent and-box are moved to the parent and-box.
The trailed bindings are promoted to the parent and-box adding wake 
entries if necessary.

\item[T4] If an and-box fails the trailed variables are unbound.



\end{itemize}

The common case where no suspension occurs now avoids many expensive
operations performed in the original scheme. Observe that tests for
stability must be modified. An and-box is unstable if the trail contains
trail entries.

\section{Choice Splitting}

Choice splitting can be done if the current and-box is stable.  An
and-box is selected as a {\em candidate}. If no candidate is found the
current and-box is suspended. The candidate is found by traversing
and-boxes in the scope of (but not including) the current and-box. An
and-box is qualified as a candidate if it is {\em solved} and the
guard operator of the and-box is a wait guard. An and-box is solved if
the sequence of tried program atoms is empty. The guard operator of
the and-box is a wait guard if the first instruction of the
and-continuation refers to a wait guard instruction.

Notice that there is no requirement that the candidate should be 
stable nor that the parent and-box is stable. The only requirement is that
the current and-box is stable.


\subsection*{Copying}

The parent choice-box of the candidate is called the {\em fork} and
the parent and-box of the fork is called the {\em mother}. The mother is the
root of the subtree to be copied, or the {\em original} subtree. A new
and-box is created which is inserted to the left of the mother. The
new and-box is the root of the {\em copy} of the subtree. The copy is
identical to the original except for the copy of the fork. The copy of the
fork has a copy of the candidate as its only and-box. The candidate is
removed from the original fork which is left with the remaining
siblings.

Copying a subtree is straightforward. Choice-boxes and and-boxes can
be copied recursively. The only problem arises when terms are copied.
When the copy is created all terms, except variables external to the
subtree, are copied. This means that new variables will be created for
the copy. A copy of a suspended and-box must constrain the copy of 
a constrained variable and the suspensions of a copy of a constrained
variable must refer to the copies of the suspended and-boxes.

When an external variable is encountered no copy should be made, but if
the variable is constrained the set of suspensions must be examined.  If
an  and-box of the original tree is suspended on the variable a new
suspension must be added for the copy of the and-box.

\subsection*{Scheduling}

Explicit scheduling information is needed to guide the execution after
a choice splitting. To maintain the scheduling information a {\em task
stack} is used. The task stack is a sequence of sets. Each set contains
the {\em task-entries} of a choice-box of the current branch.


\begin{center} \begin{tabular}{|l|}\hline
Task entry \\
\hline
\hline
and-box \\
\hline
\end{tabular} \end{center}

The choice-box above the mother is the current choice-box during the
copying procedure. If the original fork only has one remaining and-box
an entry for the deterministic and-box is added to the task stack.  If
the mother and-box is a stable and-box a task entry for the mother is
added. 

\section{The worker}


\subsection*{The execution state}

The execution state represents the information needed by a worker.


\begin{center} \begin{tabular}{|l|l|}\hline
Execution state\\
\hline
\hline
current and-box \\
current choice-box \\
argument registers\\
program counter \\
insertion point \\
trail stack \\
wake stack\\
task stack\\
\hline
\end{tabular} \end{center}

The {\em program counter} holds the next instruction to be executed.
The {\em argument registers} are used to set up the arguments of a
goal. The {\em insertion point} is the location at which the
next choice-box should be inserted. The insertion point need only be
explicitly recorded for the current and-box. The insertion point of
and-boxes in the current branch is implicitly represented by the
branch itself.

\subsection*{The execution phases}

The execution can be divided into several phases.

\begin{description}

\item[Initialize] Construct the root of the configuration.

\item[Decode] Decode instructions, perform promotions. If an inconsistent 
constraint is tried, fail the current and-box. If a guard instruction
does not allow promotion, wake suspended and-boxes.

\item[Wake] If a wake-entry is referring to a living and-box install the 
and-box.

\item[Split] If the and-box is stable and a candidate is found, perform
a choice-splitting. Install the copy of the candidate.

\item[Suspend] Undo all bindings on external variables, add constraints, 
suspensions and update stability.

\item[Schedule] If a schedule-entry refers to a living and-box, 
install the and-box. If the choice-continuation is nonempty the next
guarded goal is tried.

\item[Move up] If the choice-box is the root of the configuration,
terminate. Otherwise move to the parent and-box and set the insertion
point to the position to the right of the choice-box. Resume decoding.

\item[Install] Install each and-box in the branch from (but not including) the 
current and-box to the and-box. If the installation of an and-box
fails the and-box fails. Resume execution by decoding the first
instruction of the and-continuation.

\item[Fail] Undo all external bindings and mark the and-box as dead. 
If the parent choice-box only has one remaining and-box and the
and-continuation of the parent choice-box is empty, install the and-box.
Otherwise move to the scheduler for the parent choice-box.

\end{description}



\subsection*{Initialize}

The abstract machine is given a goal to execute. The goal is
represented by a definition and the arguments to the goal.
A root choice-box is constructed. An and-box is constructed
and added to the sequence of tried guarded goals of the root. The
and-box is the current and-box of the execution. The arguments of the
goal are constructed and stored in the argument registers.
Variables in the argument structures are all local to the and-box.

An and-continuation is added to the and-box. The permanent registers
of the and-continuation contain the variables of the arguments. The
next instruction of the and-continuation is a special guard
instruction that will never allow promotion (there is no and-box to
promote to). The insertion point is set to the first position
in the empty sequence of tried goals and the program counter is set to the
first instruction of the definition. The execution then continues by
decoding the first instruction.


\subsection*{Decoding}

When the execution of the decision instructions starts,
 the current
and-box and insertion point for the next choice-box is known.  The
program counter is pointing to the first instruction of the
definition. The decision instructions will determine a sequence of
guarded goals. If the sequence is empty the execution continues with a
failure of the current and-box.  If the sequence is nonempty a
choice-box is created and inserted at the insertion point. If the
sequence contains more than one guarded goal a choice-continuation is
created and the argument registers are saved in  the
choice-continuation. An and-box is created for the first guarded goal.
This and-box becomes the current and-box of the worker.

Instructions representing a program atom will construct the arguments
of the program atom and hand control over to the definition of the
atom.  Before control is transferred the program counter is saved in
the and-continuation. 

The instructions representing constraint atoms unify terms and
add bindings. If a unification fails the worker proceeds by failing
the current and-box.

If the guard instruction is reached all program atoms and constraint
atoms have been tried. The worker can then decide if a pruning operation
or a promotion should be performed. The conditions for each guard
instruction are found in \ref{absinstr}. 

Pruning is easy: by modifying the parent choice-box guarded goals can
effectively be removed. The and-boxes of tried guarded goals are
marked as dead and removed from the sequence. Untried guarded goals
can be removed by replacing the choice-continuation with a
null-continuation.

When an and-box is promoted the parent choice-box is removed from the
sequence of tried program atoms and the insertion point is set to the
position of the removed choice-box. The and-continuation of the
promoted and-box contains all information needed for the body of the
guarded goal.  The and-continuation is moved to the parent and-box. It
is inserted as the first and-continuation in the sequence of
and-continuations. Decoding can be resumed directly since the program
counter is referring to the first instruction of the body.

The last instruction of an and-continuation will discard the continuation. 
The execution is resumed by decoding the first instruction of the next
and-continuation. The insertion point need not be modified. 

\subsection*{Wake}
If a promotion cannot be performed the wake entries of the and-box
are examined. When all wake entries have been handled the and-box is
suspended. 

\subsection*{Schedule}
If no schedule entries exist for the parent choice-box 
the next instruction of the choice-continuation is decoded. 


\subsection*{Failing}

If an and-box fails it is marked as dead and removed from the sequence
of and-boxes. If the parent choice-box only has one remaining and-box
this and-box is tried. This must be done since the and-box may have
become leftmost or deterministic. The and-box is installed and, if the
installation succeeds, execution is resumed by decoding the first
instruction of the and-continuation. If the parent choice-box has more
than one and-box execution continues by moving up to the choice-box
and performing the scheduling tasks of the choice-box.



\section{Instructions} \label{absinstr}

The instruction set presented here is one of many possible
sets. The set presented is not very efficient, but is mainly an
adaption of the WAM instruction set. 

The instructions are divided into four groups: {\em decision},
{\em procedural}, {\em guard} and {\em term instructions}. The term
instructions are further divided into {\em get}, {\em put} and {\em
unify instructions}.

\medskip
\begin{center}\begin{tabular}{|l|l|l|} \hline
Decision instructions & Procedural instructions & Guard instructions \\
\hline
switch\_on\_term Lv La Ln Ll Ls & allocate N	& guard\_cut \\
try L 				& call P	& guard\_commit \\
retry L 			& execute P     & guard\_wait \\
trust L 			& proceed 	& guard\_halt\\
try\_single L			&		& \\
fail         			&		& \\
\hline
\end{tabular} \end{center}

\medskip
\begin{center} \begin{tabular}{|l|l|l|} \hline
Get instructions & Put instructions & Unify instructions \\
\hline
get\_variable xi/yi xi  & put\_variable xi/yi xi        & unify\_variable xi/yi \\
get\_value xi/yi xi     & put\_value xi/yi xi           & unify\_value xi/yi \\
get\_constant C xi      & put\_constant C xi            & unify\_constant xi/yi \\
get\_structure S xi     & put\_structure S xi           & unify\_structure xi/yi \\
get\_list xi            & put\_list xi                  & unify\_list xi/yi  \\
get\_nil xi             & put\_nil xi                   & unify\_nil xi/yi \\
                        & put\_void xi                  & unify\_void xi/yi \\
\hline
\end{tabular} \end{center}


\subsection*{Coding of Definitions}

An example of a definition coding is shown below.  As one can see it
is in many ways similar to a WAM \cite{wam} coding.  The coding
generation rules are not described.  An example will have to do for
the time being.

\begin{verbatim}
append([],Y,Y).
append([H|X],Y,[H|Z]):- append(X,Y,Z).
\end{verbatim}

\begin{verbatim}
append/3:
    switch_on_term Lv,La,Ln,Ll,Ls
        Lv:     try A
                trust B
        La:     try_single A
        Ln:     fail
        Ll:     try_single B
        Ls:     fail
        A:      allocate 0
                get_nil x0
                get_variable x3 x1
                get_value x3 x2
                guard_wait
                proceed
        B:      allocate 3
                get_list x0
                unify_variable x3
                unify_variable y0
                get_variable y1 x1
                get_list x2
                unify_value x3
                unify_variable y2
                guard_wait
                put_value y0 x0
                put_value y1 x1
                put_value y2 x2
                execute append/3
\end{verbatim}



\subsection*{Decision instructions}

The decision instructions perform more than just indexing; they are
also responsible for building the choice-box of the goal and the
and-box for each guarded goal. The code for a guarded goal must
therefore, unlike in the WAM, always begin with a {\em try}
instruction. If only one clause is applicable the {\em try\_single}
instruction is used.


\instruction{switch\_on\_term Lv, Ln, La, Ll, Ls}
This instruction performs indexing on the type of the argument in the
first argument register.  Given that the argument is a variable, an
integer, an atom, a list, or a structure the code is
found at addresses {\em Lv}, {\em Ln}, {\em La}, {\em Ll}, or {\em
Ls}, respectively. 

\instruction{try L} 
A choice-box and choice-continuation is created. The choice-box is
inserted at the insertion point. The arguments to the goal are stored
in the choice-continuation.  The continuation instruction is set to
the value of the program counter, i.e. the address of the instruction for
the next guarded goal.  Finally an and-box is created and inserted as
the first and-box and control is transferred to the code for the first
guarded goal.

\instruction{retry L} 
The argument registers are restored from those saved in the
choice-continu\-ation.  An and-box is created and inserted at the end of
the list of and-boxes and control is transferred to the code of the
guarded goal.

\instruction{trust L}
The trust instruction is equal to the {\em retry} instruction except that
the choice-box continuation is changed to a null-continuation.

\instruction{try\_single L} 
This instruction is used if only one clause is applicable. The
instruction is equal to the try instruction but a null-continuation is
used as a continuation.

\instruction{fail} The fail procedure is executed for the current and-box.

\subsection*{Procedural instructions}

The procedural instructions are responsible for the flow of control in 
an and-box.

\instruction{allocate N}
This instruction creates an and-continuation with room for {\em N} permanent
registers. The instruction should always follow a try instruction even for a
clause with no permanent variables since the continuation also includes the
continuation code pointer.


\instruction{call P}
The instruction initiates the calling of a goal.  The address to the
next instruction is saved in the current and-continuation. Control is
then transferred to the code of the definition.


\instruction{execute P} The instruction is different from {\em
call(L)} only in that it pertains to the last goal in the body of a
clause.  Therefore, it does not save the program counter, but instead
discards the current and-continuation. The result is the same as
executing a call instruction followed by a proceed instruction.
      
\instruction{proceed} The last instruction of a body.  The current
and-continuation is discarded, and control is transferred to code at
the address stored in the next and-continuation.


\subsection*{Guard instructions}

The guard instruction is the last instruction of the guard.

\instruction{guard\_cut }
If the and-box is quiet and leftmost the siblings of the and-box are
marked as dead and the and-box is promoted.  If the
and-box is quiet but not leftmost the and-boxes to the right of it
are removed and marked as dead.

\instruction{guard\_commit }
If the and-box is quiet the and-box is promoted. The siblings of 
the and-box are marked as dead.

\instruction{guard\_wait }
If the and-box is the last and-box it is promoted. The and-box is the
last and-box if it has not got any siblings and the parent choice box
has not got a choice-continuation.

\instruction{guard\_halt} A special guard instruction  only used in the and-box
immediately below the root choice-box. It will never allow promotion.

\subsection*{Term instructions}

The term instructions are unification specific i.e. specialized for
equality over rational trees ( {\em get\_variable}, {\em put\_value}
and {\em put\_variable} are general). 

The term instructions need an extra register called the {\em
structure pointer}. The register is used when lists and structures are
unified. The unify instructions can be executed in {\em read mode} or {\em
write mode} depending on whether a list or structure is created or not.

The arguments of the term instructions refer to the argument
registers of the execution state ({\em Xi}) and the permanent
registers ({\em Yi}) of the and-continuation.

The instructions are all quite similar to the corresponding WAM instructions. 

\subsubsection{Get instructions}

\instruction{get\_variable Xi/Yi, Xj }
The value in register Xj is moved to register Xi/Yi. 

\instruction{get\_value Xi/Yi, Xj}
The value in register Xj is unified with the value in register
Xi/Yi. 

\instruction{get\_constant C, Xj }
The constant value C is unified with the value in register Xj.

\instruction{get\_structure F, Xj }
If register Xj contains a variable, then a template for a
structure with functor F is created, and it is bound to the
variable.  Also, the structure pointer is set to refer to the first
argument of the structure, and the execution enters write mode.  If the
value in Xj is a structure, and its functor is equal to F, the
structure pointer is set to refer to the first argument of the
structure.  Otherwise, failure is performed.

\instruction{get\_list Xj }
If register Xj contains a variable, then a template for a list
cell is created, and it is bound to the variable.  Also, the structure
pointer is set to refer to its first argument, and instruction
decoding enters write mode.  If the value in Xj is a list, the
structure pointer is set to refer to the first argument of the list.
Otherwise, failure is performed.

\instruction{get\_nil Xj }
The constant value nil is unified with the value in register Xj.

\subsubsection{Unify instructions}

The unify instructions are executed in read or write mode. 

\instruction{unify\_variable Xi/Yi }
In read mode, the value referred to by the structure pointer is
moved to register Xi/Yi.  In write mode, a new variable is
created and is stored both in register Xi/Yi and in the argument
referred to by the structure pointer.  In both cases the structure
register is incremented.

\instruction{unify\_value Xi/Yi }
In read mode, the value referred to by the structure pointer is
unified with the value in register Xi/Yi.  In write mode, the
value in Xi/Yi is stored in the argument referred to by the structure
register.  In both cases the structure pointer is incremented.

\instruction{unify\_constant C}
In read mode the value referred to by the structure register is
unified with the constant C. In write mode the constant C is stored in
the argument referred to by the structure pointer. In both cases the
structure pointer is incremented.

\instruction{unify\_structure F}
In read mode the value referred to by the structure register is
unified with a template for a structure with functor F. In write mode
such a template is created  and stored in the
argument referred to by the structure pointer. In both cases the
structure pointer is set to refer to the first argument of the
structure.

\instruction{unify\_list}
In read mode the value referred to by the structure register is
unified with a template for a list structure. In write mode such a template
is created and stored in the argument referred to by
the structure pointer. In both cases the structure pointer is set to
refer to the first argument of the list.

\instruction{unify\_nil}
In read mode the value referred to by the structure register is
unified with the constant nil. In write mode the constant nil is stored in
the argument referred to by the structure pointer. In both cases the
structure pointer is incremented.

\instruction{unify\_void }
In read mode the structure register is incremented. In write mode, a new
variable is created and stored in the argument referred to by the structure
pointer. The structure pointer is then incremented.

\subsubsection{Put instructions}

\instruction{put\_variable Xi/Yi, Xj }
A new variable is created and is stored both in register Xi/Yi and in
register Xj. 

\instruction{put\_value Xi/Yi, Xj }
The value in register Xi/Yi is stored in register Xj. 

\instruction{put\_constant C, Xj }
The constant C is stored in register Xj. 

\instruction{put\_structure F, Xj } 
A template for a structure with functor F is created, and it is moved
to register Xj.  The structure register is set to refer to the first
argument of the structure, and the worker enters write mode.

\instruction{put\_list Xj }
A template for a list cell is created, and it is moved to register Xj.
The structure register is set to refer to its first argument, and the
worker enters write mode.

\instruction{put\_nil Xj }
The constant nil is stored in  register Xj.

\instruction{put\_void Xj }
A new variable is created and is stored in register Xj. 


\chapter{Implementation}\label{ch:impl}


The implementation favors flexibility over efficiency.  It is intended to be
highly portable and to serve as a test bed for further extensions. These
requirements have lead us to implement the abstract machine in terms
of an emulator. An emulator is fairly simple in its structure while still
allowing a high performance implementation.

\section{The Configuration}

While the previous chapter described the properties of the
configuration this chapter describes how the properties are explicitly
represented. 

\subsection*{Choice-boxes}

A choice-box contains a pointer to a {\em choice-continuation},
pointers to the next and the previous choice-box, a pointer to the father
and-box and a pointer to the tried and-boxes. 

\begin{verbatim}
struct choicebox {
    struct choicecont *cont;
    struct choicebox  *next, *previous;
    struct andbox     *father;
    struct predicate  *def;
    struct andbox     *tried;
    choiceboxtype     type;
}
\end{verbatim}

Choice-boxes under an and-box are doubly linked. This is done in order
to easily remove a choice-box and find the insertion point after a
promotion. 

\begin{verbatim}
struct choicecont {
    union {
       code           *label;
       bool           (*cinfo)();
    } instr;
    int               arity;
    Term              arg[ANY];
}
\end{verbatim}

A choice-continuation contains an instruction reference, the number of
argument registers and the argument registers. The instruction
reference is either a instruction pointer or a pointer to a C function.
The C function version is used to represent suspended builtin
definitions.

\subsection*{And-boxes}

The and-box contains a status field, an environment pointer, a pointer
to the first continuation, a pointer to the first unify constraint,
pointers to the next and the previous and-box, a pointer to the parent
choice-box and a pointer to the first choice-box in the tried
choice-boxes. The list of and-boxes under a choice-box is doubly
linked in order for eentries to be easily added and removed.

\begin{verbatim}
struct andbox {
    sflag              status;
    struct envid       *env;
    struct unifier     *unify;
    struct lsuspension *lsusp;
    struct choicebox   *tried;
    struct andcont     *cont;
    struct choicebox   *father;
    struct andbox      *next, *previous;
}
\end{verbatim}


The {\em status} field records whether the and-box is {\em dead}, {\em
stable} or {\em unstable}. The environment pointer is used to handle
variable environments. In a living and-box the environment pointer is
a null pointer. If the and-box is promoted it is set to point to the
environment field of the parent and-box. The use of the environment
will be explained further when variables are described.


The and-continuation contains: a instruction pointer, a pointer to
next continuation, the number of permanent register and the permanent
registers. There can be any number of permanent registers but the
number is fixed once the permanent registers have been created.

\begin{verbatim}
struct andcont {
    code              *label;
    struct andcont    *next;
    int	              ysize;
    Term              yreg[ANY];
}
\end{verbatim}

The permanent registers could be allocated on a stack. Only if the
and-box is suspended would the and-continuation be created and the registers 
saved. This would improve the performance for non-suspending programs.

\subsection*{Constraints}

A unify constraint contains the arguments of the unification. 
Unify constraints are linked in single linked lists.

\begin{verbatim}
struct unifier {
    struct unifier    *next;
    Term              var
    Term              binding;
}
\end{verbatim}

\subsection*{Terms}

Terms are represented as tagged pointers. The current tag scheme uses primary
and secondary tags in the low bits, one high GC mark bit, and integers in the
address parts of pointers. Primary tags are two bits, and need only be subtracted
in order to produce an address, since all addresses point to 4-byte aligned
objects. A secondary tag consists of two more bits above the primary tag,
which means that the tagged pointer must be shifted to the right in order to
get the address.

\begin{verbatim}
typedef unsigned long Term;
\end{verbatim}

There are seven types of terms: {\em variable}, {\em integer}, {\em atom},
{\em list}, {\em structure}, {\em float}, and {\em generic object}.

Variables use three primary tags: One for the variable reference ($REF$),
which points to a variable cell, and two more ($UVA$, $GVA$) which may be
used only within a variable cell to mark it as an unbound unconstrained or
constrained variable, respectively. The tag values are selected so that no tag
removal is necessary in order to dereference a variable, and only a single bit
test is required to distinguish a bound variable from an unbound one.

The fourth primary tag enables a secondary tag to come into play. One is used
for lists ($LST$), one for structures ($STR$), one is shared by atoms and
integers ($IMM$), and the last one is shared by floats and generic objects ($GEN$).


\begin{verbatim}
        REF           0 0
        UVA           0 1
        GVA           1 1
        LST       0 0 1 0
        IMM     x 0 1 1 0
        STR       1 0 1 0
        GEN       1 1 1 0
\end{verbatim}


\subsection*{Variables}

A variable is a pointer to a memory cell that may or may not be part of a
structure or list. As long as the variable is unbound, the contents of
the cell is a pointer with a $UVA$ or $GVA$ tag. If the tag is $UVA$, then
the pointer is an environment identifier. If the tag is $GVA$, then the
pointer points to a {\em gvainfo} structure, which consists of a pointer
to a method vector, an environment identifier and constraint-specific private
data. For AKL's builtin constraint system of rational tree equalities,
the private data consists of a pointer to a list of suspensions, and the
method pointer is $NULL$ since all methods are open-coded in this case.
This type of variable is called $SVA$ ({\em suspended variable}), when it
is needed to distinguish it from the more general case.

\begin{verbatim}
struct gvainfo {
  struct gvamethod    *method;
  struct envid        *env;
  struct suspension   *susp;            /* constraint-specific data */
} gvainfo;
\end{verbatim}

The environment of a variable points to the environment field of the
local and-box. It is set to the environment of the current and-box
when the variable is created. If a variable is local to an and-box
this can be determined simply by comparing the address of the and-box
environment with the environment of the variable.

\begin{verbatim}
struct envid {
    struct envid      *env;
}
\end{verbatim}

If the environment of the variable does not point to the environment
of the and-box, the variable environment must be dereferenced i.e.
one must follow the possibly redirected environment link until a null pointer
is reached. The dereferenced environment is then compared with the
environment of the and-box for which locality should be detected. The
dereferenced environment pointer is saved in the variable environment
to allow a simple comparison the next time the variable is used.
Notice that the most frequent case, an unpromoted local variable,
only needs a simple comparison.

\begin{verbatim}
struct suspension {
  susptag               type;
  struct suspension     *next;
  union {
    struct andbox       *andb;
    struct choicebox    *chb;
  } susp;
}
\end{verbatim}

Constrained variables have a pointer to a singly linked list of
suspensions.  Each suspension has a pointer to the and-box or
choice-box that is suspended on the variable. Suspensions are added
when an and-box is suspended or when a call to a builtin definition
suspends. This is different from the description of the abstract
machine where only and-boxes could be suspended.

When a variable becomes unconditionally bound, its memory cell is simply
overwritten with a term. For a conditional binding, the previous contents
of the memory cell must be saved in case the binding must be undone.
This is done either by value-trailing the variable, or by swapping the
contents of the memory cell with the binding in a unify constraint.


\subsection*{Atoms, integers, floats, lists and structures}

Atoms and integers share the same secondary tag. They are distinguished
by the bit next to the secondary tag, in a sense a 1-bit tertiary tag.
The address part of an atom term is a pointer to an 8-byte aligned object,
but an index into a table would work just as well.

An atom object a pointer to a print name and a saved hash value. The hash
value is used to avoid rehashing when looking up definitions. The definition
hash table is used when programs are loaded and when the {\em meta-call}
instruction is used.

An integer is represented by 26 bits stored immediately in the address part
of a tagged pointer.

A list consists of two consecutive words for the arguments.

A structure has a pointer to a functor, containing the name (an atom) and
arity, and a tuple of arguments. There is also provision for an optional
environment identifier, which may reduce the amount of copying during choice
splitting.

A generic object consist of a pointer to a method vector, followed by
object-specific data. The method vector contains pointers to procedures
to be used for copying, garbage collecting, printing, and other things.

A float is distinguished from a generic object by having a special reserved
constant in the first word, instead of a method vector.


\subsection*{Definitions}

Each definition that has been defined or that is referred to by a call
instruction is represented by a {\em definition structure}.  There are
three types of definitions: {\em emulated}, {\em builtin} and {\em
undefined}. All definitions are represented by the same structure. The
structure contains an enter instruction, name and arity, type and an
instruction pointer. The enter instruction determines the action of
the emulator when the definition is called. The instruction pointer for
an emulated definition is a pointer to the instructions of the
definition. A builtin definition has a C function pointer in the
instruction field.

\begin{verbatim}
struct definition {
    enter_instruction enter_instruction;
    int               arity;
    struct atom       *name;
    definition_type   type;
    union {
       code           *label;
       bool           (*cinfo)();
    } instr;
}
\end{verbatim}

Builtin definitions are created at boot time. The 
emulated definitions are created at load time. Undefined 
definitions are created when {\em call} instructions are loaded or
when the {\em meta call} instruction is used. 

When a {\em call} instruction is loaded the name space server is 
used to find the right definition. If the definition  
is not yet defined a undefined definition is created. When the 
definition later is defined  the undefined definition is 
changed to an emulated definition. The same thing happens if 
a {\em meta call} instruction refers to an undefined definition.

\section{The Stacks}

The worker uses five stacks to maintain its execution state. The wake,
trail and task stacks are described by the abstract machine. In
addition to these stacks a {\em recall stack} is used. The entries on
the recall stack refer to suspended choice-boxes that must be retried.
The choice-boxes were created by suspending builtin definitions.

The task entries are different from their description in the abstract
machine. There are four possible tasks {\em start}, {\em root}, {\em
stable} and {\em promote}. The stable and promote tasks are used to
guide the worker after a choice splitting. These tasks contain an
and-box pointer.  The start and root tasks have no and-box pointer.
The start task is used only when the execution is started. The root
task is used to prevent the worker from moving up from the root
choice-box. The use of the  tasks is described in \ref{impleng}.

The trail, wake, recall and task stacks are divided into segments by
{\em context entries}. The context entries are allocated on a stack,
and each choice-box/and-box pair in the current branch has a
corresponding context entry.

When the choice-box is created or entered a context entry is added to
the stack. The entry contains the current stack pointers of the trail,
wake, recall and task stacks. Entries added to the task stack will
belong to the choice-box. Entries added to the trail, wake and recall
stacks belong to the and-box below the choice-box in the current
path.

If an and-box fails the trail will be used to reset the variables. The
trail entries contain both a pointer to the variable and the value to reset
it with. The entries in the wake and recall stacks can be discarded. The task
entries will remain on the stack since these belong to the parent choice-box.
The same thing applies in a promotion: the trail is promoted but entries on
the wake and recall stack can be discarded.


\section{The Worker}

The execution state of the worker is collected in a structure. The
implementation can handle multiple execution states. Each execution state
represents an independent computation with its own configuration and
stacks. A worker may create a new execution state to encapsulate a 
computation. The worker is protected from any errors occurring in the
encapsulated computation while it can receive the results.

The execution states are linked in a tree. The first execution state
created is the root of the tree. All executions started by the root
will be linked in a singly linked list.  Each of these execution
states can have its own sub-states. The linking of execution states is
important for garbage collection and correct handling of operating
system resources.

\begin{verbatim}
struct exstate {
  struct exstate        *father;
  struct exstate        *next;
  struct exstate        *child:
  struct choicebox      *root;
  struct andbox         *andb;
  struct choicebox      *chb;
  code                  *pc;
  struct choicebox      *insert;
  struct trailhead      trail;
  struct wakehead       wake;
  struct recallhead     recall;
  struct taskhead       task;
  struct contexthead    context;
  Term                  areg[MAX_AREGS];
}
\end{verbatim}

The configuration is referred to by five pointers: root, current
and-box, current choice-box, insertion point and program counter.
These pointers are sufficient to represent the computation at
positions where we may want to save the state.


\section{The Engine}

The heart of the implementation is the engine. It consist of an
instruction emulator and procedures to carry out all operation on the
execution state. An invocation of the engine is self contained; all
parameters are local to the engine. This allows an engine to invoke an
engine recursively.


\subsection*{Initialization}

The input to the initialization is a definition and a tuple of
arguments. An execution state is created, all stacks are initialized
and a context entry is pushed on the context stack. A choice-box with
null continuation is created and added as the root of the execution
state.

Two tasks, a {\em root task} and a {\em start task} are pushed on the
task stack.  The root task works as a sentinel, and will be described
further in the following sections. The start task will be the first
task to be executed, and is only added in the initialization phase. 

An and-box is created and inserted as the only tried and-box of the
root. The and-box is called the {\em main and-box} and will be given
special attention in the execution. Its and-continuation contains the
local variables of the main goal and an instruction label refering to an
{\em guard\_halt} instruction. 

The engine is given the initialized execution state and starts the
execution. The engine will first move to the {\em choice handler}, a
procedure which decodes the entries on the task stack.  To start
the execution with a task instead of an instruction gives the freedom
to provide more complex start procedures.


\subsection*{Emulator}

The emulator reads the next instruction pointed to by the program
counter. The program counter is not updated by the read instruction.
Updating is done when the instruction is decoded, since only then do
we know the size of the instruction.

If a unification fails or if an undefined definition is called the
engine will move to the {\em fail handler}. If a guard instruction is
executed and an immediate promotion is not allowed the engine will move
to the {\em guard handler}. If the guard halt instruction is executed
the engine will move to the {\em main handler}.

\subsection*{Guard handler}

The guard handler examines the entries on the wake stack.  If a wake
entry refers to a living and-box the engine moves to the {\em wake
handler}. If the wake stack is empty  the recall stack is
examined. If a recall entry refers to a choice-box with a
continuation and a living father the engine moves to the {\em recall
handler}.

If the current and-box is not solved, marked as stable and has an
empty trail a choice splitting is tried. If a candidate is found the
engine moves to the {\em split handler} otherwise the engine moves to the
{\em suspend handler}.

\subsection*{Main handler}

The main and-box is slightly different from the general guard handler.
If all wake and recall entries have been handled and the current
and-box is solved the engine should terminate. A solution is found to
the original goal (there might be more solutions) and should be reported to 
the invoker of the engine.

If the current and-box is not solved and a choice splitting could not
be performed  the engine terminates and reports a suspended computation.

\subsection*{Wake handler}

The program counter is stored in the and-continuation of the current
and-box and the woken and-box is installed. If the installation fails
the failing and-box is made the current and-box and the engine moves to 
the fail handler.  If the installation succeeds the and-box
is made the current and-box, the program counter is set to the first 
instruction of the and-continuation and the engine moves to the emulator.


\subsection*{Recall handler}

The program counter is stored in the and-continuation of the current
and-box and the parent and-box of the woken choice-box is installed.
If the installation fails the failing and-box is made the current
and-box and execution proceed with the failure procedure.  If the
installation succeeds the and-box is made the current and-box,
Execution continues by calling the C function of the choice
continuation.

The choice continuation is set to null to stop other recall entries
from invoking the same procedure. If the builtin definition fails the 
engine moves to the fail handler. If the definition succeeds the 
program pointer is set to the first instruction of the
and-continuation and the engine moves to the emulator.


\subsection*{Split handler}

The worker moves to the choice box immediately above the mother
and-box.  After the subtree has been copied  an entry could be added to
the task stack. If the fork has only one remaining and-box a promote
task is added. Otherwise, if the mother is stable, a stable task is
added. The copy of the candidate is installed, the program counter is
set to the first instruction of the and-continuation and the engine
moves to the emulator.

\subsection*{Suspend handler}

The current and-box is suspended, the worker moves to the 
parent choice-box. The engine then moves to the choice handler.

\subsection*{Choice handler} \label{impleng}

The choice handler examines the task entries on the task stack.  The
start task is only used once to initialize registers and start the
execution. A root task will cause the engine to terminate with
failure.  Observe that by using a root task the engine does not need
to examine whether a choice-box is the root or not.

A split or promote task will cause the and-box of the task to be
installed, the program counter is set to the first instruction of the
and-continuation and the engine moves to the emulator.

By tagging the task a promote task refers to an
and-box two levels below the current choice-box. A stable and-box is
immediately below the current choice-box.

If the task stack is empty the next guarded goal of the
choice-continuation is tried. The program counter is set to the first
instruction of the continuation and the engine moves to the emulator.

If the choice continuation is empty and the list of and-boxes is empty 
the choice-box fails. The engine moves to the fail handler.

\subsection*{Fail handler}

The trailed variables and variables bound in the constraint list are
unbound. Entries on the wake and recall stacks are discarded and the
current and-box is removed. If the parent choice-box only contains a
single and-box and its continuation is empty the and-box is installed.
The program counter is set to the first instruction of the
and-continuation and the engine moves to the emulator.

If the failing and-box is the only and-box of the parent
choice-box and the choice-continuation is empty the parent and-box
fails. The problem is that the choice-box could be the root
choice-box. If it is the root there is no parent and-box to fail. 
Instead of testing whether the choice box is the root the engine
moves to the choice-handler. 

\section{Memory management}

The {\em code area} holds the instructions of definitions. An {\em
atom heap} is used to allocate atoms, functor names, definition names
and load time integers. The code area and atom heap have a fixed size,
neither is subject to garbage collection.

All run time terms are allocated on the general heap, from 
here on referred to as the {\em heap}. The heap is also used 
for the allocation of choice-boxes, choice-continuations, 
and-boxes, and-continuations, constraints and suspensions.  
The heap is automatically expanded if needed and is 
subject to garbage collection.


\subsection*{Name space servers}

Three name spaces are maintained: atom space, functor space and
definition space. Each name space is accessed through a hash table.
Each hash table holds a bucket of entries where each entry is a
pointer to an atom, a functor or a definition respectively.
The tables are fixed in size and are not subject to garbage
collection.

The atom space and functor space servers are used when new atoms and
functors are created. This happens at boot time and at load time when new
definition are defined. It can also happen at runtime if new atoms or
functors are constructed explicitly by the builtin definitions.


The definition name server is used when definitions are 
defined or referenced. This happens at boot time, at load 
time or when the meta call instruction is executed.  Ordinary 
calls do not use the name server.

\subsection*{Code area}

The code area has a static size. The default size can be 
changed by parameters given at execution time.  The area is 
not subjected to garbage collection so once we run out of 
space there is no other option but to restart the system.

Each element in the area is a 32 bit word. It holds either the op-code
of the instruction, a pointer to an atomic value, a pointer to a
definition or an index. It is clearly wasteful to use a full word in
each case but it avoids problems of alignment.


\subsection*{Constant area}

The constant area is used for structures that will not become garbage
during the execution, i.e. atoms and integers created at load time.

The constant area is not subject to garbage collection and does not
increase in size. Two things may cause the structures in the constant
area to become garbage. If a definition is redefined, structures that
are only referenced by the old definition will become unreferenced.
Also if atoms and integers defined at runtime are no longer referred to, they
become garbage.

\subsection*{Heap}

The heap is organized as a set of blocks. Each block header 
keeps a pointer to the start of the block, a pointer to the end 
of the block and a pointer to the next block.  The blocks need 
not be kept in any specific order; the addresses of items are 
never used for anything but equality.

Two lists are maintained: free blocks and used blocks.  Blocks are
removed from the list of free blocks as long as the number of used
blocks does not exceed a certain limit.  When the limit is reached a
garbage collection flag is raised. The garbage collection uses a
copying strategy where the remaining free blocks are used.

The segmented heap allows an efficient implementation of 
incremental heap space. If a larger heap is needed new blocks 
are requested from the operating system. New blocks can be 
used directly and old blocks will not be deallocated.  

The segmented heap furthermore  allows a better memory 
utilization than the fifty percent normally obtained by a 
copying garbage collector.  We can allow the system to use up 
to eighty percent of the blocks before a garbage collection is 
performed. The remaining twenty percent normally 
sufficient to hold the living data. If more blocks are needed 
new blocks are requested from the operating system.

The block size, the number of blocks and the garbage collection limit can be 
passed as parameters to the program. The only requirement is 
that the size of the blocks must not be smaller than the largest 
object. 

\subsection*{Copying} \label{implcopy}

The implementation of the copying procedure is fairly straightforward.
An and-box is copied by copying the box, the continuation, the
suspensions and the tried choice-boxes.  A choice-box is copied by
copying the box, the choice-continuation and the tried and-boxes.
When an and-box or a choice-box is copied the father pointer is set to
null and the previous pointer is set to the copy of the box. This is
done to identify a copied box and to find the copy of the box. It is
important that this is done as the first step in copying a box.

To copy the terms a scavenger algorithm is used. First a duplicate of
the term is created, and a forwarding pointer is stored in the original
structure. This allows circular structures to be copied. A pointer to
the original structure is pushed on a stack, together with the data that
was overwritten by the forwarding pointer. When the copying procedure
is finished, the original structures are restored.

The arguments of a term are scavenged with an algorithm that uses
pointer reversal to traverse the tree depth-first, right-to-left.
References to variables that have not yet been copied are registered
on a stack to be processed after all other terms have been copied. This
way, if the variable cell is part of a structure that gets copied, no extra
storage needs to be allocated for that variable.

A high mark bit is used to identify forwarding pointers. The same bit is
also used to mark the leftmost word of an argument tuple that is being
scavenged.

A constrained variable that is external to the copied subtree should
not be copied. The variable is external if it is local to an uncopied
and-box. A pointer to the the variable's gvainfo structure is pushed on
a stack. A special mark bit is set in the gvainfo structure to avoid it
from being pushed twice.

A constrained variable that is local to the subtree is copied
but the copying of the suspensions is postponed until the whole
subtree has been copied.

When the subtree has been copied the constrained external variables
are examined.  Any suspension that refers to a copied box must be
duplicated. A pointer to the copy of the box is found in its previous
field. The suspensions of constrained local variables must be copied to
the copy of the variable. The suspensions can refer to uncopied boxes
in which case they are ignored.

When the variables have been handled the forwarding pointers of the original
terms can be reset. The last step is to traverse the copied subtree and reset
all boxes.


\subsection*{Garbage collection} \label{implgc}

All execution states share the same storage allocator. This means that
garbage collection is done for all execution states at the same time.

Unused heap space is reclaimed by a stop-and-copy garbage collection.
The implementation has two advantages: the time to perform a garbage
collection is proportional to the size of the living structures and it
is easy to implement. The major disadvantage of a copying approach,
the low utilization of heap space, is largely avoided by the use of a
segmented heap.

The garbage collector uses the same algorithm as the copying
procedure. The difference is that the garbage collector does not need
to reset the copied tree. 

The garbage collection flag is examined at every {\em call}, {\em execute},
and {\em proceed} instruction, and also before suspending an and-box.


\appendix

\chapter{Definition of AKL} \label{app:akldef}

The following language definition is adapted from \cite{fra91PEPMA}.
It is included here to make this report more self-contained.  The
first few sections define the standard computation model.  In the
final section of this appendix, we introduce adaptions of the
computation model that simplify the exposition of the execution model.

\section{The Syntax of Clauses and Goals}\label{REF1}

A {\em program atom} is an atomic formula of the form
$p(x_1,\dots,x_n)$ with different variables $x_1,\dots,x_n$, called
{\em head parameters}. A {\em constraint atom} is any atomic formula
without function symbols in some constraint language. More about
constraints in section \ref{REF2}. It is assumed that these two
classes of atomic formulas are disjoint. The remaining syntactic
categories pertaining to programs are the following.
%
$$\arraycolsep 2pt%
\renewcommand{\arraystretch}{1.3}%
\begin{array}{rcl}
\nt{guarded clause}&\bnf&\nt{head}\pdef\nt{guard}\nt{guard operator}\nt{body}\\
\nt{head}&\bnf&\nt{program atom}\\
\nt{guard}&\bnf&\nt{non-empty sequence of atoms}\\
\nt{atom}&\bnf&\nt{program atom}\mid\nt{constraint atom}\\
\nt{body}&\bnf&\nt{sequence of atoms}\\
\nt{guard operator}&\bnf&\mbox{`$\wait$'}\mid\mbox{`$\cut$'}\mid\mbox{`$\commit$'}\mbox{\quad(wait, cut, commit)}
\end{array}$$%
%
A {\em definition} is a finite sequence of guarded clauses with the
same head atom and the same guard operator, defining the predicate of
the head atom. We speak of wait-definitions, cut-definitions, and
commit-definitions. Cut and commit are the {\em pruning} guard
operators. A {\em program} is a finite set of definitions where no two
definitions define the same predicate, and the predicate of every
program atom occurring in a clause in the program has a definition in
the program. The {\em local parameters} of a clause are those
variables that are not head parameters.

`\%' below stands for any of the guard operators.

The execution of an Andorra program will be represented as a series of
rewrites of goal expressions. The syntax of these expressions is
defined as follows.
%
$$\arraycolsep 2pt%
\renewcommand{\arraystretch}{1.3}%
\begin{array}{rcl}
\nt{and-box}&\bnf&\AND(\nt{non-empty sequence of local goals})_{\nt{set of variables}}\\
\nt{local goal}&\bnf&\nt{atom}\mid\nt{choice-box}\\
\nt{choice-box}&\bnf&\CHOICE(\nt{sequence of guarded goals})\\
\nt{guarded goal}&\bnf&\nt{configuration}\nt{guard operator}\nt{seq. of atoms}\\
\nt{or-box}&\bnf&\OR(\nt{sequence of configurations})\\
\nt{configuration}&\bnf&\nt{and-box} \mid \nt{or-box}\\
\nt{open goal}&\bnf&\nt{configuration}\mid\nt{local goal}\\
\nt{goal}&\bnf&\nt{open goal}\mid\nt{guarded goal}
\end{array}%
$$
%
The guarded goals in a choice-box have the same guard operator.

In the following, the letters $R$, $S$, and $T$ stand for sequences of
goals, $A$ for a sequence of atoms, and $C$ for a sequence of
constraint atoms. Any of these may stand for the empty sequence. The
letter $G$ (occasionally $X$) will be used for goals, and for the
guard in a guarded clause.  Concatenation of sequences will be written
using comma, which is not likely to be confused with the use of comma
for separating arguments.

The symbol $\FAIL$ will be used to denote the empty or-box and the
empty choice-box, regarded as collapsing to the same object. $G$ is a
{\em total failure} if $G$ is either $\FAIL$ or an or-box
$\OR(G_1,\dots,G_n)$ where each $G_i$ is a total failure.

The set of variables attached to an and-box contains the variables
{\em local} to that box, introduced in local forking or in defining an
initial goal, as described below. An and-box of the form $\AND(C)_V$
will normally be written as $C_V$.

A variable in $G$ is {\em external} to a subgoal $G'$ of $G$ if it is
local to some and-box properly containing $G'$, or if it is not local
to any and-box.

In a guarded goal, the configuration preceding the guard operator is
the {\em guard} of the goal. The atoms following the guard operator
are not regarded as goals, but become such when one of the promotion
rules defined in section \ref{REF3} is applied to the goal.

We will speak of {\em subgoals} of goals in the obvious sense. That is,
%
\begin{itemize}
\item $G$ is a subgoal of $G$,
\item The subgoals of $G_1,\dots,G_n$ are subgoals of
$\AND(G_1,\dots,G_n)_V$, $\OR(G_1,\dots,G_n)$, and
$\CHOICE(G_1,\dots,G_n)$,
\item The subgoals of $G$ are subgoals of $G \guard A$.
\end{itemize}
%
Subgoals of $G$ other than $G$ itself are called {\em proper} subgoals.

Usually, we will be talking about {\em occurrences} of subgoals,
rather than of just the syntactic form of a subgoal. So the
definitions below should in most cases be read as dealing with
occurrences of goals.  Since confusion between form and occurrence is
unlikely, the distinction will mostly be implicit.

In the case where $G$ is an and-box, an or-box, or a choice-box, we
will speak of subgoals of $G$ (including $G$ itself) as occurring {\em
in} $G$, and of $G$ as a {\em surrounding} box.  Again the
qualification ``properly'' excludes $G$ itself.

\section{Constraints}\label{REF2}

Constraints will be regarded as formulas in some constraint language.
The only constraint formulas that occur in goals are atomic
constraints without function symbols.  The letters $\sigma$, $\tau$,
and $\theta$ will be used for conjunctions of such constraints.

In the following, $\exists \sigma$ stands for the existential closure
of $\sigma$, and $\exists V$, where $V$ is the set $\{x_1,\dots,x_n\}$,
for (any permutation of) the quantifier sequence $\exists x_1 \cdots
\exists x_n$.  We allow $V$ to be empty, in which case $\exists V$ is mere
decoration.

We assume given some complete and consistent constraint theory
\mbox{\bf TC} defining the following logical properties of constraints:
%
\begin{itemize}
\item $\sigma$ is {\em satisfiable} iff $\mbox{\bf TC} \models \exists \sigma$,
\item $\theta$ is {\em unconstrained by} $\sigma$ (or: $\sigma$ {\em does not restrict} $\theta$) {\em outside} $V$ iff $$\mbox{\bf TC} \models \theta \implies\exists V (\sigma \land \theta)$$
\end{itemize}
%
This latter condition will also be expressed by saying that $\sigma$
does not restrict $\theta$ with respect to the complement of $V$.

The symbol $\TRUE$ is used to denote a variable-free atomic formula
for which it holds that $\mbox{\bf TC} \models \TRUE$. It will also be
assumed that there is a variable-free atomic formula $\FALSE$ for
which $\mbox{\bf TC} \models \neg\FALSE$. No further assumptions
concerning the properties of {\bf TC} will be made unless explicitly
stated.

That {\bf TC} is assumed to be complete is just a matter of
theoretical convenience.  Function symbols are excluded from
constraint atoms because they have no role to play in the
interpretation or execution of a program. This is no restriction from
a theoretical point of view, since any definitions of the constraint
atoms may be part of {\bf TC}. In specifying a particular language
following the AKL model we need to define not only the constraint
theory {\bf TC} but also the class of constraint atoms. In the
familiar cases of Herbrand terms (or finite trees) and rational trees,
we will speak of ``AKL with Herbrand equalities'' and ``AKL with
rational tree equalities'', meaning that the predicate $p$ of a
constraint atom either is equality, or else has a definition in the
constraint theory of the form
%
$$\forall x (p(x_1, \dots, x_n) \equiv s=t)$$
%
where s and t are terms containing no variable not in $x_1, \dots,
x_n$.  The constraint theories are in these cases {\bf HC}, the
complete elementary theory of Herbrand terms, and {\bf RC}, the
complete elementary theory of rational trees, both extended with the
above definitions.

For a sequence $R$ of goals, $\sigma(R)$ denotes the conjunction of
the constraint atoms in the sequence, or true if there are no
constraint atoms in $R$. The {\em environment} of (an occurrence of) a
subgoal of a configuration $G$, or more explicitly its environment in
$G$, is the conjunction of $\sigma(R)$ for every properly surrounding
and-box $\AND(R)_V$ in $G$, or $\TRUE$ if there is no such and-box.

Constraints $\sigma$ and $\tau$ are {\em incompatible} if
$\sigma\land\tau$ is unsatisfiable. An occurrence of $C_V$ in $G$ is
{\em quiet} if $\sigma(C)$ does not restrict the environment of $C_V$
outside $V$.

\section{Rewrite Rules}\label{REF3}

The rewrite rules below are to be understood as being applicable to any
subgoal of a goal.  That is, we define the rewrite relation on goals
%
$$G \transit G'$$
%
as meaning that $G'$ is obtainable from $G$ by rewriting one
occurrence of some subgoal $X$ of $G$ as $X'$ in accordance with one
of the rewrite rules
%
$$X \rewriteto X'$$
%
defined below. (We will say that $G \transit G'$ via $X \rewriteto
X'$.)  So references to the environment of $X$ refer to the
environment of $X$ as a subgoal of $G$.  In speaking of the goal to
which a rule is applied, we will always mean $X$ rather than $G$; an
application of a rule to a subgoal of G is an application within $G$.
A deterministic operation is an application of any rule other than
non-determinate promotion.

Note that $X$ and $X'$ are always open goals, and $X$ is never an
or-box.

\subsection*{Local Forking}

A program atom subgoal $A$ is rewritten by
%
$$A \rewriteto \CHOICE(\AND(G_1)_{V_1} \guard A_1,\dots,\AND(G_n)_{V_n} \guard A_n)$$
%
where $H \pdef G_i \guard A_i\quad(i=1,\dots,n)$ is the sequence of
clauses defining the predicate of $A$, with the arguments of $A$
substituted for the head parameters, and the local parameters of the
$i$:th clause replaced by the variables in the set $V_i$. When $G$ is
rewritten to $G'$ by applying local forking to an atomic subgoal of
$G$, the sets $V_i$ are chosen to be disjoint from the set of
variables in $G$.

\subsubsection*{Failure Propagation}

$$\AND(R,\FAIL,S)_V \rewriteto \FAIL$$

\subsubsection*{Choice Elimination}

$$\CHOICE(R,\FAIL\guard A,S) \rewriteto \CHOICE(R,S)$$

\subsubsection*{Environment Synchronization}

$$\AND(R)_V \rewriteto \FAIL$$ 
%
if $\sigma(R)$ is incompatible with the environment of the and-box.

\subsubsection*{Guard Distribution}

$$\CHOICE(R,\OR(G,S) \guard A,T) \rewriteto \CHOICE(R,G \guard A,\OR(S) \guard A,T)$$

\subsubsection*{Determinate Promotion}

$$\AND(R,\CHOICE(C_V \guard A),S)_W \rewriteto \AND(R,C,A,S)_{V \cup W}$$
%
if $C_V$ is satisfiable, and in case $\guard$ is cut or commit is also
quiet.  The rewrite is performed {\em with respect to} $C_V$, and also
{\em with respect to} $\CHOICE(C_V \guard A)$.

\subsubsection*{Cut}

$$\CHOICE(R,C_V \cut A,S) \rewriteto \CHOICE(R,C_V \cut A)$$
%
if $S$ is non-empty and $C_V$ is satisfiable and quiet.

\subsubsection*{Commit}

$$\CHOICE(R,C_V \commit A,S) \rewriteto \CHOICE(C_V \commit A)$$
%
if $R$ or $S$ is non-empty and $C_V$ is satisfiable and quiet.

\subsubsection*{Non-Determinate Promotion}

{\renewcommand{\arraystretch}{1.3}%
\begin{eqnarray*}
\lefteqn{\AND(T_1, \CHOICE(R, C_V \wait A, S), T_2)_W \rewriteto}\\
\mbox{\qquad}& &\OR(\AND(T_1, C, A, T_2)_{V\cup W}, \AND(T_1, \CHOICE(R, S), T_2)_W)
\end{eqnarray*}}%
%
if $R$ or $S$ is non-empty. The condition for non-determinate
promotion to be applicable is a bit complicated and merits separate
treatment.

\section{Stability and Non-Determinate Promotion}\label{REF4}

A couple of definitions are useful in discussing non-determinate
promotion: a cut-guard or commit-guard in $G$ of the form $C_V$ is
{\em suspended} if it is not quiet. A step {\em unfreezes} a suspended
guard if the guard is quiet after that step. An application of cut,
commit, or environment synchronization is {\em non-trivial} if the
environment of the goal to which the operation is applied is
consistent. Other operations are always non-trivial.

In an application of non-determinate promotion as exhibited above we
will say that the rule is applied {\em with respect} to the guard
$C_V$.  Also, an and-box of the form $$\AND(T_1, \CHOICE(R, C_V \wait
A, S), T_2)_W$$ where $R$ or $S$ is non-empty will be called a {\em
candidate} for non-determinate promotion.  We can refer to
$C_V$ as the {\em candidate solution}.

The basic Andorra principle of performing determinate promotion in
preference to non-determinate promotion translates into a necessary
condition for applying non-determinate promotion, viz. that no
deterministic operation is applicable to or within the and-box. This
will be called the non-determinacy condition.

In AKL, the basic Andorra principle is given a stronger
interpretation: non-determinate promotion is to be delayed until (i)
the non-determinacy condition is satisfied, and (ii) non-trivial
deterministic operations on or within the and-box cannot become
possible as a result of any future changes in the environment of the
and-box through deterministic operations. This will be called the AKL
principle.

The AKL principle will be seen to be a consequence of the restrictions
on non-determinate promotion now to be formulated. We distinguish
between the AKL principle and the particular restrictions to be
described here, since these restrictions, although guided by the AKL
principle, also incorporate specific ideas concerning what is useful
in programming. We need the concept of {\em stability}: an and-box $G$
which is a subgoal of an and-box $G'$ is stable (relative to $G'$) if
\begin{itemize}
\item[i)] no deterministic operation is applicable to or within $G$, and
\item[ii)] no series of rewrites of $G'$, in which no restriction is imposed
on non-deter\-min\-ate promotion apart from the non-determinacy
condition, can lead to a goal in which a non-trivial deterministic
operation is applicable to or within $G$.
\end{itemize}
In speaking of stable and-boxes below, we will mean boxes stable
relative to the whole configuration being rewritten.  Note that the
property of stability is well-defined (although in general
undecidable) since this definition does not invoke any further
restrictions on non-determinate promotion.

The conditions on non-determinate promotion can now be stated:
non-deter\-min\-ate promotion is applicable to an and-box $$G =
\AND(T_1, \CHOICE(R, C_V \wait A, S), T_2)_W$$ if and only if $G$
has a nearest surrounding stable choice-box $G'$ such that every
candidate for non-determinate promotion in $G'$ has $G'$ as its
nearest surrounding stable and-box.

Given these restrictions, it easily follows that the AKL principle
will be satisfied. For no deterministic operation is applicable to or
within a subgoal of a stable and-box, and if $G$ is a subgoal of a stable
and-box, further rewrites may enable non-trivial deterministic operations
within $G$ only as a result of applications of non-determinate promotion
to surrounding and-boxes.

Additional conditions may be imposed on non-determinate promotion,
stipulating e.g. that non-determinate promotion must be applied
with respect to the leftmost possible solved guard, or that it must
be applied to an innermost eligible candidate.

In an implementation of AKL one will need to replace the abstract
definition of stability by a more managable sufficient condition for
stability. Of course the part of the definition stipulating that no
deterministic operation be applicable to $G$ is non-problematic.  What
we need to consider are sufficient conditions for ruling out any
future non-trivial application of the rules for cut, commit, and
environment synchronization.

To deal with environment synchronization, we can give a condition that
is independent of the constraint theory TC. It is easily shown that
any future non-trivial environment synchronization in G can be ruled
out if the following {\em environment condition} holds:
\begin{quote}
For every and-box $\AND(R)_V$ in $G$ with environment $\tau$ in $G$,
$\sigma(R) \land \tau$ does not restrict the environment of $G$ with
respect to the variables external to $G$.
\end{quote}

We also need to formulate a {\em suspension stability condition}, that
is, a condition that will ensure that no future deterministic step
will unfreeze any suspended guard in $G$, unless the environment of
$G$ becomes inconsistent. The environment condition is not in general
itself a suspension stability condition. For a counterexample, suppose
$G$ has external variables $y_1$ and $y_2$, on which the environment
of $G$ imposes no condition, and a suspended constraint $x<y_1$, where
the environment of the constraint in $G$ is $x<y_2$.  Assuming $<$ to
refer to real numbers, the conjunction of these does not restrict the
environment of $G$, but if $y_2<y_1$ is added to that environment,
$x<y_1$ will become quiet.  Similar counterexamples apply if the
constraints include equalities in the field of real numbers, or
inequalities between Herbrand terms. These examples suggest that a
suspension stability condition that covers many different constraint
theories must be rather stringent. One sufficient condition is clearly
that there are in $G$ no suspended constraints containing external
variables.

In the special cases of Herbrand equalities and rational tree
equalities it turns out that the environment condition is in fact
itself a suspension stability condition. These two types of
constraints seem to be the only naturally occurring ones for which
this holds.

\section{Computations}

The first thing to note, by an inspection of the rules and of the
syntax of goals, is that rewriting a goal in accordance with one of
the rules does in fact result in another goal.  Thus, for example,
replacing an occurrence of an and-box in a goal by an occurrence of an
or-box yields a new goal.

An {\em initial} goal has the form $\AND(A)_V$, with $V$ a subset of
the set of variables occurring in any of the atoms in the non-empty
sequence $A$ (the existentially quantified variables in the initial
goal). We define a {\em computation} as a finite or infinite sequence
$G_1,G_2,\dots$ of goals, where $G_1$ is an initial goal and $G_i
\transit G_{i+1}$ for every non-final $i$. A derivation
$G_1',G_2',\dots$ is a {\em subderivation} of $G_1,G_2,\dots$ if each
$G_i'$ is a subgoal of $G_{k_i}$, for some subsequence
$G_{k_1},G_{k_2},\dots$ of $G_1,G_2,\dots$.  We will use the notation
$G \transit^* G'$ to mean that there is a computation in which $G=G_i$
and $G'=G_j$ for some $i<j$.

It is easily seen that all goals occurring in a computation are
and-boxes or or-boxes, that is, configurations. In particular, all
configurations after the first top-level application of
non-deterministic promotion (if there is such) will be or-boxes, in
general nested, with no flattening taking place, since there is no
rule for flattening or-boxes. Or-boxes occurring in guards, on the
other hand, may be flattened in the course of a computation.  (By
using the guard distribution rule in combination with the choice
elimination rule we can also reduce $\OR(G)$ to $G$ and $\OR(R,\FAIL)$
to $\OR(R)$.)

\section{Adaptations for the Execution Model}
\label{sec:akldefadapt}

Here we adapt the standard computation model, described in the
previous sections, for the use as a basis for an execution model.

\subsection{No Or-Boxes}

The top-level box will always be a choice-box, instead of an and-box
or an or-box, and the non-determinate promotion rule is replaced by
the new choice splitting rule.  The guard distribution rule is no
longer needed.  The reason for this change is twofold.  First, in the
execution model, certain operations that are associated with promotion
(determinate and non-determinate) will only have to be performed by
determinate promotion.  Second, we rid ourselves of or-boxes
altogether.

In the execution model, we do not use the non-determinate promotion
rule as is.  Instead, we use the {\em choice splitting rule}, defined
as follows.
%
% TeX is high-level type-setting language...
%
$$\vbox{\hbox{$\CHOICE(R_1, \AND(S_1, \CHOICE(T_1, C_V\wait A, T_2), S_2)_W \guard B, R_2) \rewriteto$}\vskip2mm\hbox{\hskip1cm\vbox{\hbox{$\CHOICE(R_1, \AND(S_1, \CHOICE(C_V \wait A), S_2)_W \guard B,$}\hbox{\setbox0=\hbox{$\CHOICE(R_1, \hbox{}$}\hskip\wd0$\AND(S_1, \CHOICE(T_1, T_2), S_2)_W \guard B, R_2)$}}}}$$
%
The conditions for choice splitting are the derived from those for
non-determinate promotion.  The sequence $T_1$ or the sequence $T_2$
has to be non-empty.  The conditions for its applicability are derived
from those for non-determinate promotion, but the terminology is
slightly different.

In a choice-box of the form%
%
$$\CHOICE(R_1,\AND(S_1, \CHOICE(T_1, C_V \wait A, T_2), S_2)_W \guard B, R_2)$$
%
the solved and-box $C_V$ is called a {\em candidate} for choice
splitting, and the choice splitting rule can be applied {\em with
respect to} the candidate $C_V$.  The and-box and the choice-box
matching
%
$$\AND(S_1, \CHOICE(T_1, C_V \wait A, T_2), S_2)_W$$
%
are called the {\em copied} boxes.  After the application of choice
splitting, the and-box and the choice-box matching
%
$$\AND(S_1, \CHOICE(C_V \wait A), S_2)_W$$
%
are called the {\em left} copies and the and-box and the choice-box
matching
%
$$\AND(S_1, \CHOICE(T_1, T_2), S_2)_W$$
%
are called the {\em right} copies.

Choice splitting can be applied to a choice-box $G$ with respect to a
candidate $C_V$ if and only if $C_V$ has a nearest properly
surrounding stable and-box $G'$ such that every candidate for choice
splitting in $G'$ has $G'$ as its nearest properly surrounding stable
and-box.

Putting this slightly more intuitively, we can say that given a stable
and-box $G$, we may apply choice splitting with respect to a candidate
in $G$, if there is no stable and-box $G'$ in $G$ in which we can
apply choice splitting.  We should choose the innermost stable
and-box.  In the execution model, this condition is slightly relaxed.
We will refer to the worker's ability to {\em establish stability}.
The worker should choose the innermost and-box for which it is able to
establish stability.%  In section \ref{sec:stability} are found
%different schemes for establishing stability, one of which is
%complete.

We also add the condition that we should choose the {\em leftmost}
candidate within the stable and-box.

The effect of the choice splitting rule and the non-determinate
promotion rule can be related as follows.  If an and-box satisfies the
conditions for non-determinate promotion, and it occurs in a
choice-box, this choice-box satisfies the conditions for choice
splitting.  After applying non-determinate promotion on this and-box,
subsequently applying guard distribution twice on the branches of the
new or-box, and finally guard elimination on the remainder, the
situation is the same as after applying choice splitting, followed by
determinate promotion of the left copy of the copied choice-box.

As modication of the formal language definition, we state that the
top-level box should be a choice-box.  Therefore, an {\em initial
goal} is here of the form
%
$$\CHOICE(\AND(G_1, \dots, G_n)_V \wait \TRUE)$$
%
As can be easily verified, no applications of rules eliminate the
top-level choice-box, thus all goals in a computation will have a
choice-box as the topmost goal, and no or-boxes will be created.


\section*{Acknowledgements}

We thank the members of the Andorra group at SICS: Kent Boortz, Per
Brand, and Torkel Franz\'en.  We would also like to thank Johan
Bevemyr, who wrote the first versions of the emulator and compiler in
three months in the summer of 1990.  No-one could have done it
quicker.

This work was part of ESPRIT Project 2471 (``PEPMA''), in which SICS'
efforts were supported by the Swedish National Board for Industrial
and Technical Development (NUTEK), Swedish Telecom, and Ericsson
Group.  We thank the other members of this project for their
contributions to this work.

SICS is a non-profit research foundation, sponsored by the Swedish
National Board for Industrial and Technical Development (NUTEK), Asea
Brown Boveri AB, Ericsson Group, IBM Svenska AB, NobelTech Systems AB,
the Swedish Defence Material Administration (FMV), and Swedish
Telecom.

\bibliographystyle{plain}
\bibliography{akl}

\end{document}
