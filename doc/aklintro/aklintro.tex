\documentstyle[11pt,a4]{article}

%%%
%%% Some definitions
%%%

\def\progf{}

\begingroup
\catcode`\|=13
\catcode`\_=13
\catcode`\<=13
\catcode`\>=13

\gdef\prog{\progf%
\catcode`\|=13%
\def|{$\mid$}%
\catcode`\_=13%
\def_{\ifmmode\else\_\fi}%
\catcode`\<=13%
\def<{\ifmmode\char`\<\else$<$\fi}%
\catcode`\>=13%
\def>{\ifmmode\char`\>\else$>$\fi}%
}
\endgroup

\newenvironment{program}{%
\begingroup%
\prog%
\obeylines%
\def\nl{\\[\medskipamount]}%
\begin{tabbing}%
\hskip1cm\=\hskip1cm\=\hskip1cm\=\hskip1cm\=\hskip1cm\=\hskip1cm\=\kill}{%
\end{tabbing}\endgroup\noindent}

\newenvironment{progex}{%
\begingroup%
\prog%
\def\nl{\\[\medskipamount]}%
\begin{tabbing}%
\hskip1cm\hskip1cm\=\+\hskip1cm\=\hskip1cm\=\hskip1cm\=\hskip1cm\=\kill}{%
\end{tabbing}\endgroup\noindent}

\def\cond{$\rightarrow$}

\def\nonterm#1{\langle\mbox{\em #1\/}\rangle}
\def\nt#1{\ifmmode\nonterm{#1}\else$\nonterm{#1}$\fi}

%%%
%%% Main body of text
%%%

\title{An Introduction to AKL\\
A Multi-Paradigm Programming Language}

\author{Sverker Janson and Seif Haridi\\
Swedish Institute of Computer Science\\ Box 1263, S-164 28 KISTA,
Sweden\\ E-mail sverker@sics.se, seif@sics.se}

\begin{document}

\maketitle

\section{Introduction}

AKL is a multi-paradigm programming language based on a concurrent
constraint framework \cite{jaha91}, directly or indirectly supporting
the following paradigms.
%
\begin{itemize}
\item	processes and process communication,
\item	object-oriented programming,
\item	functional and relational programming,
\item	constraint programming.
\end{itemize}
%
These aspects of AKL are cleanly integrated, and provided using a
minimum of basic concepts, common to them all.  AKL agents will serve
as processes, objects, functions, relations, or constraints, depending
on the context.

AKL is a programming language kernel.  Some aspects of a complete
programming language, a user language, have been omitted, such as type
declarations and modules, a standard library, and direct syntactic
support for some of the programming paradigms; but the programming
paradigms and the basic implementation technology developed for AKL
will carry over to any user language based on AKL.
                     
% \verb|Figure 3.1  Multifaceted AKL agents|

In the following sections, we will introduce AKL, then describe
process programming in AKL, object-oriented programming in AKL,
functional and relational programming in AKL, and constraint
programming in AKL.  Finally, it will be shown how these aspects may
be integrated in an application.

\section{Concurrent Constraint Programming}

AKL is based on the concept of concurrent constraint programming, a 
paradigm distinguished by its elegant notions of communication and 
synchronisation based on constraints \cite{sar93}.

In a concurrent constraint programming language, a computation state
consists of a group of {\em agents} and a {\em store} that they share.
Agents may add pieces of information to the store, an operation called
{\em telling}, and may also wait for the presence in the store of
pieces of information, an operation called {\em asking}.  The
information in the store is expressed in terms of {\em constraints},
which are statements in some constraint language, usually based on
first-order logic, e.g.,
%
\begin{progex}
X < 1,  Y = Z + X,  W = [a, b, c], \dots
\end{progex}%
%
If telling makes a store inconsistent, the computation fails (more on
this later).  Asking a constraint means waiting until the asked
constraint either is {\em entailed} by (follows logically from) the
information accumulated in the store or is {\em disentailed} by (the
negation follows logically from) the same information.  In other
words, no action is taken until it has been established that the asked
constraint is true or false.  For example, {\prog X < 1} is obviously
entailed by {\prog X~=~0} and disentailed by {\prog X~=~1}.

Constraints restrict the range of possible values of variables that
are shared between agents.  A variable may be thought of as a
container.  Whereas variables in conventional languages hold single
values, variables in concurrent constraint programming languages may
be thought of as holding the (possibly infinite) set of values
consistent with the constraints currently in the store.  This
extensional view may be complemented by an intensional view, in which
each variable is thought of as holding the constraints which restrict
it.  This latter view is often more useful as a mental model.
                                  
\begin{figure}[ht]
  \vspace{4.5cm}
  \centerline{\special{agents.ps}}
  \caption{Agents interacting with a constraint store}\label{fig:agents}
\end{figure}

% \verb|Figure 2.1  Agents interacting with a constraint store|

The range of constraints that may be used in a program is defined by
the current {\em constraint system}, which in AKL, in principle, may
be any first-order theory.  In practice, it is necessary to ensure
that the telling and asking operations used are computable and have a
reasonable computational complexity.  Constraint systems as such are
not discussed here.  For the purpose of this introduction, we will use
a simple constraint system with a few obvious constraints, which is
essentially that of Prolog and GHC to which arithmetic has been added.

Thus, constraints in AKL will be formulas of the form
%
\begin{progex}
\nt{expression} = \nt{expression} \\
\nt{expression} $\neq$ \nt{expression} \\
\nt{expression} < \nt{expression}
\end{progex}%
%
and the like.  Equality constraints, e.g., {\prog X~=~1}, are often
called {\em bindings}, suggesting that the variable {\prog X} is {\em
bound} to {\prog 1} by the constraint.  Correspondingly, the act of
telling a binding on a variable is called {\em binding} the variable.
Expressions are either {\em variables} (alpha-numeric symbols with an
upper case initial letter), e.g.,
%
\begin{progex}
X, Y, Z, X$_1$, Y$_1$, Z$_1$, \dots
\end{progex}%
or {\em numbers}, e.g.,
%
\begin{progex}
1, 3.1415, -42, \dots
\end{progex}%
%
or {\em arithmetic expressions}, e.g.,
%
\begin{progex}
1 + X, -Y, X * Y, \dots
\end{progex}%
%
or {\em constants}, e.g.,
%
\begin{progex}
a, b, c, \dots
\end{progex}%
%
or {\em constructor expressions} of the form
%
\begin{progex}
\nt{name}(\nt{expression}, \dots, \nt{expression})
\end{progex}%
%
where \nt{name} is an alpha-numeric symbol with a lower case initial
letter, e.g.,
%
\begin{progex}
s(s(0)), tree(X, L, R), \dots
\end{progex}%
%
There is also the constant {\prog []}, which denotes the empty list,
and the list constructor {\prog [\nt{expression}|\nt{expression}]}.  A
syntactic convention used in the following is that, e.g., the
expression {\prog [a|[b|[c|d]]]} may be written as {\prog
[a,\,b,\,c|d]}, and the expression {\prog [a|[b|[c|[]]]]} may be
written as {\prog [a,\,b,\,c]}.  In addition we assume that
constraints {\em true} and {\em false} are available, which are
independent of the constraint system and may be identified with their
corresponding logical constants.


\section{Basic Concepts}

The agents of concurrent constraint programming correspond to
statements being executed concurrently.  Constraints, as described in
the previous section, are atomic statements known as {\em constraint
atoms} (or just constraints).  When they are asked and when they are
told is discussed in the following.

A {\em program atom} statement of the form
%
\begin{progex}
\nt{name}(X$_1$, \dots, X$_n$)
\end{progex}%
%
is a defined agent.  In a program atom, \nt{name} is an alpha-numeric
symbol and $n$ is the arity of the atom.  The variables {\prog X$_1$,
\dots, X$_n$} are the {\em actual parameters} of the atom.  Occurrences of
program atoms in programs are sometimes referred to as {\em calls}.
Atoms of the above form may be referred to as {\prog \nt{name}/$n$}
atoms, e.g.,
%
\begin{progex}
plus(X, Y, Z)
\end{progex}%
%
is a {\prog plus/3} atom.  Occasionally, when no ambiguity can arise,
``$/n$'' is dropped.  The behaviour of atoms is given by {\em (agent)
definitions} of the form
%
\begin{progex}
\nt{name}(X$_1$, \dots, X$_n$) := \nt{statement}.
\end{progex}%
%
The variables {\prog X$_1$, \dots, X$_n$} must be different and are
called {\em formal parameters}.  During execution, any atom matching
the left hand side will be replaced by the statement on the right hand
side, with actual parameters replacing occurrences of the formal
parameters.  A definition of the above form is said to define the
{\prog \nt{name}/$n$} atom, e.g.,
%
\begin{progex}
plus(X, Y, Z) := Z = X + Y.
\end{progex}%
%
is a definition of {\prog plus/3}.

A {\em composition} statement of the form
%
\begin{progex}
\nt{statement}, \dots, \nt{statement}
\end{progex}%
%
builds a composite agent from a sequence of agents.  Its behaviour is
to replace itself with the concurrently executing agents corresponding
to its components.

A {\em conditional choice} statement of the form
%
\begin{progex}
(\nt{statement} \cond\ \nt{statement} ; \nt{statement})
\end{progex}%
%
is used to express conditional execution.  Let us call its components
condition, then-branch, and else-branch, respectively.  (Later a more
general version of this statement will be introduced.)

Let us, for simplicity, assume that the condition is a constraint.  A
conditional choice statement will ask the constraint in the condition
from the store.  If it is entailed, the then-branch replaces the
statement.  If it is disentailed, the else-branch replaces the
statement.  If neither, the statement will wait until either becomes
known.  If the condition is an arbitrary statement, the above
described actions will take place when the condition has been reduced
to a constraint or when it fails.  The concept of failure is discussed
later.

A {\em hiding} statement of the form
%
\begin{progex}
X$_1$, \dots, X$_n$ : \nt{statement}
\end{progex}%
%
introduces variables with local scope.  The behaviour of a hiding
statement is to replace itself with its component statement, in which
the variables {\prog X$_1$, \dots, X$_n$} have been replaced by new
variables.

Let us at this point establish some syntactic conventions.
%
\begin{itemize}
\item
Composition binds tighter than hiding, e.g.,
%
\begin{progex}
X : p, q, r
\end{progex}%
%
means
%
\begin{progex}
X : (p, q, r)
\end{progex}%
%
Parentheses may be used to override this default, e.g.,
%
\begin{progex}
(X : p), q, r
\end{progex}%
%
\item
Any variable occurring free in a definition (i.e., not as one of the
formal parameters, nor introduced by a hiding statement) is implicitly
introduced by a hiding statement enclosing the right hand side of the
definition, e.g.,
%
\begin{progex}
p(X, Y) := q(X, Z), r(Z, Y).
\end{progex}%
%
where Z occurs free, means
%
\begin{progex}
p(X, Y) := Z : q(X, Z), r(Z, Y).
\end{progex}%
%
in which hiding has been made explicit.

\item
Expressions may be used as arguments to program atoms, and will then
correspond to bindings on the actual parameters, e.g.,
%
\begin{progex}
p(X+1, [a,\,b,\,c])
\end{progex}%
%
means
%
\begin{progex}
( Y, Z : Y = X+1, Z = [a,\,b,\,c], p(Y, Z) )
\end{progex}%
%
where the new arguments have also been made local by hiding.
\end{itemize}
%
It is now time for a first small example: an append/3 agent which is
used to concatenate two lists.
%
\begin{program}
append(X, Y, Z) := \\
\>\>( X = [] \cond\ Z = Y \\
\>\>; X = [E|X$_1$], append(X$_1$, Y, Z$_1$),  Z = [E|Z$_1$] ).
\end{program}%
%
It will initially suffice to think about constraints in two different
ways, depending on the context in which they occur.  When occurring as
conditions, constraints are asked.  Elsewhere, they are told.

In append/3, the condition {\prog X~=~[]} is asked, which means that
it may be read ``as usual''.  If it is entailed, the then-branch is
chosen, in which {\prog Z~=~Y} is told.  If the condition is
disentailed, the else-branch is chosen.  There, {\prog X~=~[E|X$_1$]}
is told.  Since {\prog X} is not {\prog []}, it is assumed that it is
a list constructor, in which E is equal to the head of {\prog X} and
{\prog X$_1$} equal to the tail of {\prog X.}  The recursive append
call makes {\prog Z$_1$} the concatenation of {\prog X$_1$} and {\prog
Y.}  The final constraint {\prog Z~=~[E|Z$_1$]} builds the output
{\prog Z} from {\prog E} and the partial result {\prog Z$_1$}.

Note how variables allow us to work with incomplete data.  In a call
%
\begin{progex}
append([1,\,2,\,3], Y, Z)
\end{progex}%
%
the parameters {\prog Z} and {\prog Y} can be left unconstrained.  The
third parameter {\prog Z} may still be computed as {\prog
[1,\,2,\,3|Y]}, where the tail {\prog Y} is unconstrained.  If {\prog
Y} is later constrained by, e.g., {\prog Y~=~[]}, then it is also the
case that {\prog Z~=~[1,\,2,\,3]}.

Variables are also indirectly the means of communication and
synchronisation.  If a constraint on a variable is asked, the
corresponding agent, e.g., conditional choice statement, is suspended
and may be restarted whenever an appropriate constraint is told on the
variable by another agent.

At this point it seems appropriate to illustrate the nature of
concurrent computation in AKL.  The following definitions will create a
list of numbers, and add together a list of numbers, respectively.
%
\begin{program}
list(N, L) := \\
\>\>( N = 0 \cond\ L = [] \\
\>\>; L = [N|L$_1$], list(N -- 1, L$_1$) ).  \nl
sum(L, N) := \\
\>\>( L = [] \cond\ N = 0 \\
\>\>; L = [M|L$_1$], sum(L$_1$, N$_1$), N = N$_1$ + M ).
\end{program}%
%
The following computation is possible.  In the examples, computations
will be shown by performing rewriting steps on the state (or
configuration) at hand, unfolding definitions and substituting values
for variables, etc., where appropriate, which should be intuitive.  In
this example we avoid details by showing only the relevant atoms and
the collection of constraints on the output variable {\prog N}.
Intermediate computation steps are skipped.  Thus,
%
\begin{progex}
list(3, L), sum(L, N)
\end{progex}%
%
is rewritten to
%
\begin{progex}
list(2, L$_1$), sum([3|L$_1$], N)
\end{progex}%
%
by unfolding the {\prog list} atom, executing the choice statement,
and substituting values for variables according to equality
constraints.  This result may in its turn be rewritten to
%
\begin{progex}
list(1, L$_2$), sum([2|L$_2$], N$_1$), N = 3 + N$_1$
\end{progex}%
%
by similar manipulations of the {\prog list} and {\prog sum} atoms.
Further possible states are
%
\begin{progex}
list(0, L$_3$), sum([1|L$_3$], N$_2$), N = 5 + N$_2$ \\
sum([], N$_3$), N = 6 + N$_3$ \\
N = 6
\end{progex}%
%
with final state N = 6.

The {\prog list/2} call produces a list, and the {\prog sum/2} call is
there to consume its parts as soon as they are created.  The logical
variable allows the {\prog sum/2} call to know when data has arrived.
If the tail of the list being consumed by the {\prog sum/2} call is
unconstrained, the {\prog sum/2} call will wait for it to be produced
(in this case by the {\prog list/2} call).

In this example, a particular execution order was chosen, but observe
that the final result is quite independent of the execution order.

The simple set of constructs introduced so far is a fairly complete
programming language in itself, quite comparable in expressive power
to, e.g., functional programming languages.  If we were merely looking
for Turing completeness, the language could be restricted, and the
constraint systems could be weakened considerably.  But then important
aspects such as concurrency, modularity, and, of course,
expressiveness would all be sacrificed on the altar of simplicity.

In the following sections, we will introduce constructs that address
the specific needs of important programming paradigms, such as
processes and process communication, object-oriented programming,
relational programming, and constraint satisfaction.  In particular, we
will need the ability to choose between alternative computations in a
manner more flexible than that provided by conditional choice.

\section{Don't Care Nondeterminism}

In concurrent programming, processes should be able to react to
incoming communication from different sources.  In constraint
programming, constraint propagating agents should be able to react to
different conditions.  Both of these cases can be expressed as a number
of possibly non-exclusive conditions with corresponding branches.  If
one condition is satisfied, its branch is chosen.

For this, AKL provides the {\em committed choice} statement
%
\begin{progex}
( \nt{statement} | \nt{statement} \\
; \dots \\
; \nt{statement} | \nt{statement} )
\end{progex}%
%
The symbol ``{\prog |}'' is called {\em commit}.  The statement
preceding commit is called a {\em guard} and the statement following
it is called a {\em body}.  A pair
%
\begin{progex}
\nt{statement} | \nt{statement}
\end{progex}%
%
is called a {\em (guarded) clause}, and may be enclosed in hiding as
follows.
%
\begin{progex}
X$_1$, \dots, X$_n$ : \nt{statement} | \nt{statement}
\end{progex}%
%
The variables {\prog X$_1$}, \dots, {\prog X$_n$} are called {\em
local variables} of the clause.

Let us first, for simplicity, assume that the guards are all
constraints.  The commit\-ted-choice statement will ask all guards
from the store.  If any of the guards is entailed, the composition of
its constraint and its corresponding body replaces the
committed-choice statement.  If a guard is disentailed, its
corresponding clause is deleted.  If all clauses are deleted, the
committed choice statement fails.  Otherwise, it will wait.  Thus, it
may select an arbitrary entailed guard, and commit the computation to
its corresponding body.

If a variable Y is hidden, an asked constraint is preceded by the
expression ``for some {\prog Y}'' (or logically, ``{\prog
$\exists$Y}'').  For example, in
%
\begin{progex}
X = f(a), ( Y : X = f(Y) | q(Y) )
\end{progex}%
%
the asked constraint is {\prog $\exists$Y(X~=~f(Y))} (``for some
{\prog Y}, {\prog X~=~f(Y)}''), which is entailed, since there exists
a {\prog Y} (namely ``{\prog a}'') such that {\prog X~=~f(Y)} is
entailed.

List merging may now be expressed as follows, as an example of an agent 
receiving input from two different sources.
%
\begin{program}
merge(X, Y, Z) := \\
\>\>( X = [] | Z = Y \\
\>\>; Y = [] | Z = X \\
\>\>; E, X$_1$ : X = [E|X$_1$] | Z = [E|Z$_1$], merge(X$_1$, Y, Z$_1$) \\
\>\>; E, Y$_1$ : Y = [E|Y$_1$] | Z = [E|Z$_1$], merge(X, Y$_1$, Z$_1$) ).
\end{program}%
%
A merge agent can react as soon as either {\prog X} or {\prog Y} is
given a value.  In the last two clauses, hiding introduces variables
that are used for ``matching'' in the guard, as discussed above.
These variables are constrained to be equal to the corresponding list
components.

\section{Don't Know Nondeterminism}

Many problems, especially frequent in the field of Artifical
Intelligence, and also found elsewhere, e.g., in operations research,
are currently solvable only by resorting to some form of search.  Many
of these admit very concise solutions if the programming language
abstracts away the details of search by providing don't know
nondeterminism.

For this, AKL provides the {\em nondeterminate choice} (or {\em don't
know choice}) statement.
%
\begin{progex}
( \nt{statement} ? \nt{statement} \\
; \dots \\
; \nt{statement} ? \nt{statement} )
\end{progex}%
%
The symbol ``{\prog ?}'' is called {\em wait}.  The statement is
otherwise like the committed choice statement in that its components
are called {\em (guarded) clauses}, the components of a clause {\em
guard} and {\em body}, and a clause may be enclosed in hiding.

Again we assume that the guards are all constraints.  The
nondeterminate choice statement will also ask all guards from the
store.  If a guard is disentailed, its corresponding clause is
deleted.  If all clauses are deleted, the choice statement fails.  If
only one clause remains, the choice statement is said to be
determinate.  Then the composition of the constraint in the remaining
guard and its corresponding body replaces the choice statement.
Otherwise, if there is more than one clause left, the choice statement
will wait.  Subsequent telling of other agents may make it
determinate.  If eventually a state is reached in which no other
computation step is possible, each of the remaining clauses may be
tried in different copies of the state.  The alternative computation
paths are explored concurrently.

Let us first consider a very simple example, an agent that accepts
either of the constants a or b, and then does nothing.
%
\begin{program}
p(X) := \\
\>\>( X = a ? true \\
\>\>; X = b ? true ).
\end{program}%
%
The interesting thing happens when the agent {\prog p} is called with
an unconstrained variable as an argument.  That is, we expect it to
produce output.  Let us call {\prog p} together with an agent {\prog
q} examining the output of {\prog p}.
%
\begin{program}
q(X, Y) := \\
\>\>( X = a \cond\ Y = 1 \\
\>\>; Y = 0 ).
\end{program}%
%
Then the following is one possible computation starting from
%
\begin{progex}
p(X), q(X, Y)
\end{progex}%
%
First p and q are both unfolded.
%
\begin{progex}
( X = a ? true ; X = b ? true ), ( X = a \cond\ Y = 1 ; Y = 0 )
\end{progex}%
%
At this point in the computation, the nondeterminate choice statement
is nondeterminate, and the conditional choice statement cannot
establish the truth or falsity of its condition.  The computation can
now only proceed by trying the clauses of the nondeterminate choice in
different copies of the computation state.  Thus,
%
\begin{progex}
X = a, ( X = a \cond\ Y = 1 ; Y = 0 ) \\
Y = 1
\end{progex}%
%
and
%
\begin{progex}
X = b, ( X = a \cond\ Y = 1 ; Y = 0 ) \\
Y = 0
\end{progex}%
%
are the two possible computations.  Observe that the nondeterminate
alternatives are ordered in the order of the clauses in the
nondeterminate choice statement.  This ordering will be used later.

Now, what could possibly be the use of having an agent generate
alternative results? This we will try to answer in the following.  It
will help to think of the alternative results as a sequence of
results.  Composition of two agents will compute the intersection of
the two sequences of results.  This will be illustrated using the
member agent, which examines membership in a list.
%
\begin{program}
member(X, Y) := \\
\>\>( Y$_1$ : Y = [X|Y$_1$] ? true \\
\>\>; X$_1$, Y$_1$ : Y = [X$_1$|Y$_1$] ? member(X, Y$_1$) ).
\end{program}%
%
The agent
%
\begin{progex}
member(X, [a,\,b,\,c])
\end{progex}%
%
will establish whether the value of {\prog X} is in the list {\prog
[a,\,b,\,c]}.  When the agent is called with an unconstrained {\prog
X}, the different members of the list are returned as different
possible results (in the order {\prog a, b, c}, due to the way the
program is written).  The composition
%
\begin{progex}
member(X, [a,\,b,\,c]), member(X, [b,\,c,\,d])
\end{progex}%
%
will compute the {\prog X} that are members in both lists.  When two
nondeterminate choice statements are available, the leftmost is
chosen.  In this case it will enumerate members of the first list,
creating three alternative states
%
\begin{progex}
X = a, member(X, [b,\,c,\,d]) \\
X = b, member(X, [b,\,c,\,d]) \\
X = c, member(X, [b,\,c,\,d])
\end{progex}%
%
The members in the first list that are not members in the second are
eliminated by the failure of the corresponding alternative
computations.  A computation that fails leaves no trace in the sequence
of results, and the two final alternative states will be
%
\begin{progex}
X = b \\
X = c 
\end{progex}%
%
In fact, the sequence of results may become empty, as in the case of
the following composition
%
\begin{progex}
member(X, [a,\,b,\,c]), member(X, [d,\,e,\,f])
\end{progex}%
%
Such complete failure is also useful, as discussed in the following.

\section{General Statements in Guards}

Although we have ignored it up to this point, any statement may be
used as a guard in a choice statement.  The behaviour presented above
has been that of the special case when conditions and guards are
constraints.  This will now be generalised.

Before we proceed, we introduce the general conditional choice statement.
%
\begin{progex}
( \nt{statement} \cond\ \nt{statement} \\
; \dots \\
; \nt{statement} \cond\ \nt{statement} )
\end{progex}%
%
The symbol ``{\prog \cond}'' is called {\em then}.  Again, the
statement is otherwise like the other choice statements in that its
components are called {\em (guarded) clauses}, the components of a
clause {\em guard} and {\em body}, and a clause may be enclosed in
hiding.

The previously introduced version of conditional choice is, of course,
merely syntactic sugar for the special case
%
\begin{progex}
( \nt{statement} \cond\ \nt{statement} \\
; true \cond\ \nt{statement} )
\end{progex}%
%
The case where the guard of the last clause is ``{\prog true}'' is
common enough to warrant general syntactic sugar, thus
%
\begin{progex}
( \nt{statement} \cond\ \nt{statement} \\
; \dots \\
; true \cond\ \nt{statement} )
\end{progex}%
%
may always be abbreviated to
%
\begin{progex}
( \nt{statement} \cond\ \nt{statement} \\
; \dots \\
; \nt{statement} )
\end{progex}%
%
For the last time we make the simplifying assumption that the guards
are all constraints.  The conditional choice statement asks the
constraint of the first guard.  If it is entailed, the composition of
it and its body replaces the choice statement.  If it is disentailed,
the clause is deleted, and the next clause is tried.  If neither, the
statement will wait.  These steps are repeated as necessary.  If no
clauses remain, the conditional choice statement fails.

When a more general statement is used as a guard, it will first be
executed locally in the guard, reducing itself to a constraint, after
which the previously described actions take place.  To illustrate this
before we descend into the details, let us use append in a guard (a
fairly unusual guard though).
%
\begin{progex}
( append(X, Y, Z) \cond\ p(Z) \\
; true \cond\ q(X, Y) )
\end{progex}%
%
If we supply constraints for {\prog X} and {\prog Y}, e.g., {\prog
X~=~[1]}, {\prog Y~=~[2,\,3]}, a value will be computed locally for
{\prog Z}, and the resulting choice statement is
%
\begin{progex}
( Z = [1,\,2,\,3] \cond\ p(Z) \\
; true \cond\ q(X, Y) )
\end{progex}%
%
with its above described behaviour.

Formally, the computation in the guard is a separate computation, with
local agents and its own local constraint store.  Constraints told by
local agents are placed in the local store, but constraints asked by
local agents are asked from the union of the local store and external
stores.  Locally told constraints can thus be observed by local agents,
but not by agents external to the guard.
                                             
% \verb|Figure 2.2: An agent with a local constraint store|

When the local computation terminates successfully, the constraint
asked for the guard is the conjunction of constraints in its local
constraint store.  This coincides with the behaviour in the special
case that the guard was a constraint.  In fact, the behaviour of a
constraint atom statement is always to tell its constraint to the
current constraint store.

If the local store becomes inconsistent with the union of external
stores, the local computation fails.  The behaviour is then as if the
computation had terminated successfully, its constraint had been
asked, and it had been found disentailed by the external stores.

The scope of don't know nondeterminism in a guard is limited to its
corresponding clause.  New alternative computations for a guard will be
introduced as new alternative clauses.  This will be illustrated using
the following simple nondeterminate agent.
%
\begin{program}
one_or_one(X, Y) := \\
\>\>( X = 1 ? true \\
\>\>; Y = 1 ? true ).
\end{program}%
%
Let us start with the statement
%
\begin{progex}
( one_or_one(X, Y) | q )
\end{progex}%
%
The {\prog one_or_one} atom is unfolded, giving
%
\begin{progex}
( ( X = 1 ? true ; Y = 1 ? true ) | q )
\end{progex}%
%
Since no other step is possible, we may try the alternatives of the
nondeterminate choice in different copies of the closest enclosing
clause, which is duplicated as follows.
%
\begin{progex}
( X = 1 | q \\
; Y = 1 | q )
\end{progex}%
%
Other choice statements are handled analogously.

Before leaving the subject of don't know nondeterminism in guards, it
should be clarified exactly when alternatives may be tried.  A
(possibly local) state with agents and their store is {\em (locally)
stable} if no computation step other than copying in nondeterminate
choice is possible, and no such computation step can be made possible
by adding constraints to external constraint stores (if any).
Alternatives may be tried for the leftmost possible nondeterminate
choice in a stable state.

By only executing a nondeterminate choice in a stable state, don't
know nondeterministic computations will be synchronised in a
concurrent setting in a manner not unlike the synchronisation achieved
by conditional or committed choice.  For example, the agent
%
\begin{progex}
member(X, Y)
\end{progex}%
%
will unfold to
%
\begin{progex}
( Y$_1$ : Y = [X|Y$_1$] ? true \\
; X$_1$, Y$_1$ : Y = [X$_1$|Y$_1$] ? member(X, Y$_1$) )
\end{progex}%
%
By adding constraints to the environment of this agent, it is possible
to continue execution without copying, e.g., by adding {\prog X~=~1}
and {\prog Y~=~[2|W]}.  Thus, while there are active agents in its
environment that may potentially tell constraints on {\prog Y}, the above
agent is unstable.

\section{Bagof}

Finally, we introduce a statement which builds lists of sequences of
alternative results.  It provides powerful means of interaction
between determinate and nondeterminate code.  It is similar to the
corresponding construct in Prolog, and a generalisation of the list
comprehension primitive found in functional languages (e.g., Haskell).

A {\em bagof} statement of the form
%
\begin{progex}
bagof(\nt{variable}, \nt{statement}, \nt{variable})
\end{progex}%
%
builds a list of the sequence of alternative results from its
component statement.  The different alternative bindings for the
variable in the first argument will be collected as a list in the
variable in the last argument.  The statement will be executed within
the bagof statement in a manner not unlike the execution of a
guard.  Don't know nondeterminism is not propagated outside it.

For example, the composition
%
\begin{progex}
member(X, [a,\,b,\,c]), member(X, [b,\,c,\,d])
\end{progex}%
%
has two alternative results {\prog X~=~b} and {\prog X~=~c}.  By
wrapping this composition in a bagof statement, collecting different
alternatives for {\prog X} in {\prog Y}
%
\begin{progex}
bagof(X, ( member(X, [a,\,b,\,c]), member(X, [b,\,c,\,d]) ), Y)
\end{progex}%
%
the result becomes
%
\begin{progex}
Y = [b,\,c]
\end{progex}%
%
as could be expected.  Bagof exists in two varieties: ordered (the
default) and unordered.  The don't know nondeterministic alternatives
are, as usual, ordered in the order of clauses in the nondeterminate
choice.  Thus,
%
\begin{progex}
((X = a ; X = b) ; (X = c ; X = d))
\end{progex}%
%
generates alternatives for X in the order a, b, c, d.  So,
%
\begin{progex}
bagof(X, ((X = a ; X = b) ; (X = c ; X = d)), Y)
\end{progex}%
%
yields {\prog Y~=~[a,\,b,\,c,\,d]}.  However,
%
\begin{progex}
unordered_bagof(X, ((X = a ; X = b) ; (X = c ; X = d)), Y)
\end{progex}%
%
ignores this order, and collects an alternative in the list as soon as
it is available.  Depending on the implementation, this could lead to
a different order, e.g., {\prog Y~=~[d,\,c,\,b,\,a]}.

\section{More Syntactic Sugar}

Analogously to what is usually done for functional languages, we now
introduce syntactic sugar that is convenient when the guards in choice
statements consist mainly of pattern matching against the arguments,
as is often the case.

A definition of the form
%
\begin{program}
p(X$_1$, \dots, X$_n$) := \\
\>\>( g$_1$ \% b$_1$ \\
\>\>; \dots \\
\>\>; g$_k$ \% b$_k$ ).
\end{program}%
%
where {\prog \%} is either {\prog \cond}, {\prog |}, or {\prog ?}, may
be broken up into several clauses
%
\begin{program}
p(X$_1$, \dots, X$_n$) :-- g$_1$ \% b$_1$.  \\
\dots \\
p(X$_1$, \dots, X$_n$) :-- g$_k$ \% b$_k$.
\end{program}%
%
which together stand for the above definition.

The main point of this transformation into clausal definitions is that
the following additional syntactic sugar may be introduced, which will
be exemplified below: (1) Free variables are implicitly hidden, but
here the hiding statement encloses the right hand side of the clause
(i.e., to the right of ``{\prog :--}''), and not the entire
definition.  (2) Equality constraints on the arguments in the guard
part of a clause may be folded back into the heads {\prog p(X1, \dots,
Xn)} of these clauses.  (3) If the remainder of the guard is the null
statement ``{\prog true}'', it may be omitted.  (4) If the guard is
omitted and the guard operator is wait ``{\prog ?}'', it may also be
omitted.  (5) If the guard operator is omitted, and the body is the
null statement ``{\prog true}'', a clause may be abbreviated to a
head.

As an example, the definition
%
\begin{program}
member(X, Y) := \\
\>\>( Y$_1$ : Y = [X|Y$_1$] ? true \\
\>\>; X$_1$, Y$_1$ : Y = [X$_1$|Y$_1$] ? member(X, Y1) ).
\end{program}%
%
may be transformed to clauses
%
\begin{program}
member(X, Y) :-- \\
\> \>Y = [X|Y$_1$] \\
\>?\>true.  \\
member(X, Y) :-- \\
\> \>Y = [X$_1$|Y$_1$] \\
\>?\>member(X, Y$_1$).
\end{program}%
%
where hiding is implicit according to (1).  The equality constraints
may then be folded back into the head according to (2), and the
remaining null guards may be omitted according to (3), giving
%
\begin{program}
member(X, [X|Y$_1$]) :-- \\
\>?\>true.  \nl
member(X, [X$_1$|Y$_1$]) :-- \\
\>?\>member(X, Y$_1$).
\end{program}%
%
which may be further abbreviated to
%
\begin{program}
member(X, [X|Y$_1$]).  \\
member(X, [X$_1$|Y$_1$]) :-- \\
\>\>member(X, Y$_1$).
\end{program}%
%
according to (4) and (5).  We exemplify also with the append and merge
definitions.
%
\begin{program}
append([], Y, Z) :-- \\
\>\cond\ \>Y = Z.  \\
append(X, Y, X) :-- \\
\>\cond\ \>X = [E|X$_1$], \\
\>	\>Z = [E|Z$_1$], \\
\>	\>append(X$_1$, Y, Z$_1$).\nl
merge([], Y, Z) :-- \\
\>|\>Y = Z.  \\
merge(X, [], Z) :-- \\
\>|\>X = Z.  \\
merge([E|X], Y, Z) :-- \\
\>|\>Z = [E|Z$_1$], \\
\> \>merge(X, Y, Z$_1$).  \\
merge(X, [E|Y], Z) :-- \\
\>|\>Z = [E|Z$_1$], \\
\> \>merge(X, Y, Z$_1$).
\end{program}%
%
The examples should make it clear that some additional clarity is
gained with the clausal syntax, which prevails in the logic
programming community.  We end this section with a few additional
remarks about the syntax.

As syntactic sugar, the underscore symbol ``{\prog _}'' may be used in
place of a variable that has a single occurrence in a clause.  All
occurrences of ``{\prog _}'' in a definition denote different
variables.

In an implementation of AKL, the character set restricts our
syntax.  The then symbol ``{\prog \cond}'' is there written as ``{\prog
->}'', and subscripted indices are not possible.  For example, append
would be written as
%
\begin{program}
append([], Y, Z) :-- \\
\>->\>Y = Z.  \\
append(X, Y, Z) :-- \\
\>->\>X = [E|X1], \\
\>  \>append(X1, Y, Z1), \\
\>  \>Z = [E|Z1].
\end{program}%
%
which is a program that can be compiled and run in the AKL Programming
System \cite{jamo92}.  However, to make programs as readable as
possible, we will continue to use ``{\prog \cond}'' and indices.


\section{Processes and Process Communication}

Agents may be thought of as processes, and telling constraints on
shared variables may be thought of as communicating on a shared
channel.  The basic principles supporting the idea of communicating
processes were discussed in the previous sections.  Here we will
expand the discussion by explaining many of the concurrent programming
idioms.  These are inherited from concurrent logic programming (see,
e.g., \cite{sha87}).

\subsection{Communication and Streams}

The underlying idea is that a logical variable may be used as a {\em
communication channel}.  On this channel, a message can be sent by a
producer process by binding the variable to some value.
%
\begin{progex}
X = a
\end{progex}%
%
A conditional or a committed-choice statement may be used by a
consumer process to achieve the effect of waiting for a message.  By
imposing suitable constraints on the communication variable in their
guards, these statements will require the value of the variable to be
defined before execution may proceed.  Until the value has been
produced, the statement will be suspended.
%
\begin{progex}
( X = a | this \\
; X = b | that )
\end{progex}%
%
However, as soon as the variable is constrained, the guard parts of
these statements may be executed, and the appropriate action can be
taken.  Message arguments can be transferred by binding the variable
to a constructor expression.
%
\begin{progex}
X = f(Y)
\end{progex}%
%
Likewise, the argument can be received by matching against a
constructor expression.
%
\begin{progex}
( Y : X = f(Y) | this(Y) \\
; Y : X = g(Y) | that(Y) )
\end{progex}%
%
Again, note the scope of the hiding statement.  It is limited to each
guarded statement.  If {\prog Y} were given a wider scope, the first
guard would instead be that the value of {\prog X} should be equal to
{\prog X~=~f(Y)}, for some given value of {\prog Y}.  The above use
has the reading ``if there exists a {\prog Y} such that {\prog
X~=~f(Y)} \dots'', and it allows {\prog Y} to be constrained by the
guard.

Contrary to what is the case in the above examples, communication is
not restricted to a single message between a producer and a
consumer.  A message can be given an argument that is the variable on
which the next message will be sent.  Usually, the list constructor is
used for this purpose.  The first argument of the list constructor is
the message, and the second argument is the new variable.  A sequence
of messages ({\prog M$_i$}) can be sent as follows.
%
\begin{progex}
X$_0$ = [M$_1$|X$_1$], X$_1$ = [M$_2$|X$_2$], X$_2$ = [M$_3$|X$_3$], \dots
\end{progex}%
%
The receiver waits for a list constructor, and expects the message to
arrive in the first argument, and the variable on which further
messages will be sent in the second argument.  Observe that the above
example is simply the construction of a list of messages.  When used
to transfer a sequence of messages between processes, a list is
referred to as a {\em stream}.  Just like a list, a stream may end
with {\prog []}, which indicates that the stream has been closed, and
that no further messages will be sent.

Understood in these terms, the list-sum example above is a typical
{\em pro\-ducer-con\-sumer} example.  The list agent produces a stream
of messages, each of which is a number, and the sum agent consumes the
stream, adding the numbers together.

\subsection{Basic Stream Techniques}

In the previous section, we discussed the notions of producers and
consumers.  The list-agent is an example of a producer, and the
sum-agent is an example of a consumer.  Further basic stream techniques
are stream transducers, distributors, and mergers.
                                          
\begin{figure}[ht]
  \vspace{4.8cm}
  \centerline{\special{procidiom.ps}}
  \caption{Transducer, distributor, and merger}\label{fig:procidiom}
\end{figure}

% \verb|Figure 3.2  Transducer, distributor, and merger|

A stream {\em transducer} is an agent that takes one stream as input
and produces another stream as output.  This may involve computing new
messages from old, rearranging, deleting, or adding messages.  The
following is a simple stream transducer computing the square of each
incoming message.
%
\begin{program}
squares([], Out) :-- \\
\>\cond\>Out = [].  \\
squares([N|Ns], Out) :-- \\
\>\cond\>Out = [N*N|Out1], \\
\>\>squares(Ns, Out1).
\end{program}%
%
A stream {\em distributor} is an agent with one input stream and
several output streams that directs incoming messages to the
appropriate output stream.  The following is a simple stream
distributor that sends apples to one stream and oranges to the other.
%
\begin{program}
fruits([], As, Os) :-- \\
\>\cond\>As = [], \\
\>\>Os = [].  \\
fruits([F|Fs], As, Os) :-- \\
\>\>	apple(F) \\
\>\cond\>As = [F|As$_1$], \\
\>\>	fruits(Fs, As$_1$, Os).  \\
fruits([F|Fs], As, Os) :-- \\
\>\>	orange(F) \\
\>\cond\>	Os = [F|Os$_1$], \\
\>\>	fruits(Fs, As, Os$_1$).
\end{program}%
%
A stream {\em merger} is an agent with several input streams and one
output stream that interleaves messages from the input streams into
the single output stream.  The following is the standard binary stream
merger, which was also shown in the language introduction.
%
\begin{program}
merge([], Ys, Zs) :-- \\
\>|\>	Zs = Ys.  \\
merge(Xs, [], Zs) :-- \\
\>|\>	Zs = Xs.  \\
merge([X|Xs], Ys, Zs) :-- \\
\>|\>	Zs = [X|Zs$_1$], \\
\>\>	merge(Xs, Ys, Zs$_1$).  \\
merge(Xs, [Y|Ys], Zs) :-- \\
\>|\>	Zs = [Y|Zs$_1$], \\
\>\>	merge(Xs, Ys, Zs$_1$).
\end{program}%
%
Note that all the above definitions can also be seen as simple
list-processing agents.  However, they are more interesting when one
considers their behaviour as components in concurrent programs.

\subsection{Process Structures}

Process networks can be used for storing data.  This is an example of
an object\-oriented reading of processes.  The technique is best
introduced by an example.  We will show how a dictionary can be
represented as a binary tree of processes.
                                                     
\begin{figure}[ht]
  \vspace{5.6cm}
  \centerline{\special{proctree.ps}}
  \caption{Tree of node and leaf processes}\label{fig:proctree}
\end{figure}

% \verb|Figure 3.3  Tree of node and leaf processes|

The tree is built from {\prog leaf} processes and {\prog node}
processes.  A {\prog leaf} process has one input stream from its
parent.  A {\prog node} process has one input stream from its parent
and two output streams to its children.  In addition, it has two
arguments for holding the key and the value of the data item stored in
the node.

Thus, the processes correspond to equivalent data-structures.  In
their default state, these processes are waiting for messages on their
input streams.  The messages may be of the kind {\prog insert(Key,
Value)}, with given key and value that should be inserted, {\prog
lookup(Key, Result)}, with a given key and a sought for result (an
unconstrained variable), and the closing of the stream which means
that the tree should terminate (deallocate itself).

This technique, to include a variable in the message for the return
value, is common enough to warrant a name of its own: {\em incomplete
messages}.

The computed result is wrapped in the constructor {\prog
found(Value)}, if a value corresponding to a key is found, and is
otherwise the constant ``{\prog not_found}''.  When a {\prog node}
process receives a request, it compares the key to the key held in its
argument, and either takes care of the request itself, or passes the
request along to its left or right sub-tree, depending on the result
of the comparison.  A {\prog leaf} process always processes a request
itself.

%
\begin{program}
dict(S) := leaf(S).\nl
leaf([]) :-- \\
\>\cond\>	true.  \\
leaf([insert(K,V)|S]) :-- \\
\>\cond\>	node(S, K, V, L, R), \\
\>\>	leaf(L), \\
\>\>	leaf(R).  \\
leaf([lookup(K,V)|S]) :-- \\
\>\cond\>	V = not_found, \\
\>\>	leaf(S).  \nl
node([], _, _, L, R) :-- \\
\>\cond\>	L = [], \\
\>\>	R = [].  \\
node([insert(K1, V1)|S], K, V, L, R) :-- \\
\>\cond\>	(\>\>K1 = K  \\
\>\>		\>\cond\ \>node(S, K, V1, L, R) \\
\>\>	;	\>\>K1 < K \\
\>\>		\>\cond\>L = [insert(K1, V1)|L1], \\
\>\>		\>\>node(S, K, V, L1, R) \\
\>\>	;	\>\>K1 > K \\
\>\>		\>\cond\ \>R = [insert(K1, V1)|R1], \\
\>\>		\>\>node(S, K, V, L, R1) ).  \\
node([lookup(K1, V1)|S], K, V, L, R) :-- \\
\>\cond\>	(\>\>K1 = K \\
\>\>		\>\cond\ \>V1 = found(V), \\
\>\>		\>\>node(S, K, V, L, R) \\
\>\>	;	\>\>K1 < K \\
\>\>		\>\cond\ \>L = [lookup(K1, V1)|L1], \\
\>\>		\>\>node(S, K, V, L1, R) \\
\>\>	;	\>\>K1 > K \\
\>\>		\>\cond\ \>R = [lookup(K1, V1)|R1], \\
\>\>		\>\>node(S, K, V, L, R1) ).
\end{program}%
%
In the following section on object-oriented programming, we will
relate this programming technique to conventional object-oriented
programming and its standard terminology.

\section{Object-Oriented Programming}

In this section, the basic techniques that allow us to do
object-oriented programming in AKL are reviewed.  Like the programming
techniques in the previous section, they belong to logic programming
folklore.

There is more than one way to map the abstract concept of an object
onto corresponding concepts in a concurrent constraint language.  The
first and most wide\-spread of these will be described here in detail
\cite{st83}.  It is based on the process reading of
logic programs.  Several embedded languages have been proposed that
support this style of programming (e.g., \cite{vulcan,aum,polka}).
They are typically much less verbose, and they also provide more
explicit support for objectoriented concepts.

As will be seen, in this framework there is no real need for an
implementation of objects, unlike the case when one is adding object
oriented support to a language such as C.  Following an object-oriented
style of programming is a very natural thing.

\subsection{Objects}

An {\em object} is an abstract entity that provides services to its
clients.  Clients explicitly request services from objects.  The
request identifies the requested service, as well as the objects that
are to perform the service.

Objects are realised as processes that take as input a stream (a list)
of requests.  The stream identifies the object.  The data associated
with the objects are held in the arguments of the process.  An object
definition typically has one clause per type of request, which
performs the corresponding service, and one clause for terminating (or
deallocating) the object.  Thus, clauses correspond to methods.

\begin{figure}[ht]
  \vspace{5.009cm}
  \centerline{\special{setting.ps}}
  \caption{An object consuming a list of messages}\label{fig:setting}
\end{figure}

The requests are typically expressions of the form {\prog name(A, B,
C)}, where the constructor ``{\prog name}'' identifies the request,
and {\prog A}, {\prog B}, and {\prog C} are the arguments of the
request.

The process description, the agent definition, is the class, the
implementation of the object.  The individual calls to this agent are
the instances.  A standard example of an object is the bank account,
providing withdrawal, deposits, etc.
%
\begin{program}
make_bank_account(S) := \\
\>\>	bank_account(S, 0).  \nl
bank_account([], _) :-- \\
\>\cond\>	true.  \\
bank_account([withdraw(A)|R], N) :-- \\
\>\cond\>	bank_account(R, N -- A).  \\
bank_account([deposit(A)|R], N) :-- \\
\>\cond\>	bank_account(R, N + A).  \\
bank_account([balance(M)|R], N) :-- \\
\>\cond\>	M = N, \\
\>\>	bank_account(R, N).
\end{program}%
%
A computation starting with
%
\begin{progex}
make_bank_account(S), \\
S = [balance(B$_1$), deposit(7), withdraw(3), balance(B$_2$)]
\end{progex}%
%
yields
%
\begin{progex}
B$_1$ = 0, B$_2$ = 4
\end{progex}%
%
A bank-account object is created by starting a process {\prog
bank_account(S, 0)} given as initial input an unspecified stream
{\prog S} (a variable) and a zero balance.  The stream {\prog S} is
used to identify the object.  A service {\prog deposit(5)} is
requested by binding {\prog S} to {\prog [deposit(5)|S$_1$]}.  The
next request is added to {\prog S$_1$}, and so on.  In the above
example, only one clause will match any given request.  When it is
applied, some computation is performed in its body and a new {\prog
bank_account} process replaces the original one.  The requests in the
above example are processed as follows.  Let us start in the middle.
%
\begin{progex}
bank_account(S, 0), S = [deposit(7), withdraw(3), balance(B$_2$)].
\end{progex}%
%
The {\prog bank_account} process is reduced by the clause matching the
first deposit request, leaving some computation to be performed.
%
\begin{progex}
N = 0+7, bank_account(S$_1$, N), S$_1$ = [withdraw(3), balance(B$_2$)].
\end{progex}%
%
This leaves us with.
%
\begin{progex}
bank_account(S, 7), S$_1$ = [withdraw(3), balance(B$_2$)].
\end{progex}%
%
The rest of the requests are processed similarly.

Finally, there are a few things to note about these objects.  First,
they are automatically encapsulated.  Clients are prevented from
directly accessing the data associated with an object.  In imperative
languages, this is not as self-evident, as the object is often
confused with the storage used to store its internal data, and the
object identifier is a pointer to this storage, which may often be
used for any purpose.

Second, requests are entirely generic.  The expression that identifies
a request may be interpreted differently, and may therefore involve
the execution of different code, depending on the object.  This does
not involve mandatory declarations in some shared (abstract or
virtual) ancestor class, as in many other languages.

Third, becoming another type of object is extremely simple.  Instead of
replacing itself with an object of the same type, an object may pass
its stream, and appropriate parameters, on to a new object.  An example
of this was given in the section on process structures, where a leaf
process became a node process when a message was inserted into a
binary tree.

\subsection{Inheritance}

In the object-oriented paradigm, objects can be classified in terms of
the services they provide.  One object may provide a subset of the
services of another object.  This way an interface hierarchy is formed.

It is of course important, from a software engineering point of view,
that the descriptions of objects higher up in the hierarchy can be
reused as parts of the descendant objects.  This is either done by
inheritance or by delegation.  Delegation is easily achieved in the
framework we describe.  However, since requests are completely
generic, it is also possible to design an interface hierarchy without
inheritance or delegation, if so desired.

Delegation is achieved by creating instances of the ancestor
objects.  The object identifier of (the stream to) this ancestor object
is held as an argument of the derived object.  The object corresponding
to the ancestor could appropriately be called a subobject of the
derived object.  The derived object filters incoming requests and
delegates unknown requests to its subobject.

Delegation is not restricted to unknown requests.  We may also define
what is elsewhere known as after- and before-methods by filtering as
well.  The derived object may perform any action before passing a
request on to a subobject.

Let us derive from the {\prog bank_account} class a kind of account
that does some form of logging of incoming requests.  Let us say that
it also adds a {\prog get_log} service that returns the log.  This is
easy.
%
\begin{program}
make_logging_account(S) := \\
\>\>	make_bank_account(O), \\
\>\>	make_empty_log(Log), \\
\>\>	logging_account(S, O, Log).  \nl
logging_account([get_log(L)|R], O, Log) :-- \\
\>\cond\>	L = Log, \\
\>\>	logging_account(R, O, Log).  \\
logging_account([Req|R], O, Log) :-- \\
\>\cond\>	O = [Req|O$_1$], \\
\>\>	add_to_log(Req, Log, Log$_1$), \\
\>\>	logging_account(R, O$_1$, Log$_1$).  \\
logging_account([], O, _) :-- \\
\>\cond\>	O = [].
\end{program}%
%
With delegation, it is cumbersome to handle the notion of self
correctly.  Modern forms of multiple inheritance, based on the
principle of specialisation, are also difficult to achieve.  Instead,
it is quite possible to view inheritance as providing the ability to
share common portions of object definitions by placing them in
super-classes, which are then implicitly copied into sub-class
definitions.  To exploit this view, syntactic support has to be added
to the language, e.g., along the lines of Goldberg and Shapiro
\cite{sha92}.  This view corresponds closely to that of conventional
object-oriented languages.

\subsection{Ports for Objects}

{\em Ports} are a special form of constraints, which, when added to
AKL, or to any concurrent logic programming language, will solve a
number of problems with the approach to object-oriented programming
presented above, problems that we have avoided mentioning so far.
This section provides a preliminary introduction to ports.  They, and
the problems they solve, are described in great detail elsewhere
\cite{jamoha93}.

A port is a binary constraint on a bag (a multi-set) of messages and a
corresponding stream of these messages.  It simply states that they
contain the same messages, in any order.  A bag connected to a stream
by a port is usually identified with the port, and is referred to as a
port.  The {\prog open_port(P, S)} operation relates a bag {\prog P}
to a stream {\prog S}, and connects them through a port.  The stream
{\prog S} will usually be connected to an object.  Instead of using
the stream to access the object, we will send messages by adding them
to the port.  The {\prog send(M, P)} operation sends a message {\prog
M} to a port {\prog P}.  To satisfy the port constraint, a message
sent to a port will immediately be added to its associated stream,
first come first served.

When a port is no longer referenced from other parts of the
computation state, when it becomes garbage, it is assumed that it
contains no more messages, and its associated stream is automatically
closed.  When the stream is closed, any object consuming it is thereby
notified that there are no more clients requesting its services.

Thus, to summarise: A port is created with an associated stream (to an
object).  Messages are sent to the port, and appear on the stream in
any order.  When the port is no longer in use, the stream is closed,
and the object may choose to terminate.

A simple example follows.
%
\begin{progex}
open_port(P, S), send(a, P), send(b, P)
\end{progex}%
%
yields
%
\begin{progex}
P = $\nt{a port}$, S = [a,\,b]
\end{progex}%
%
Here we create a port and a related stream, and send two messages.  The
messages appear in {\prog S} in the order of the send operations in the
composition, but it could just as well have been reversed.  The stream
is closed when the messages have been sent, since there are no more
references to the port.

Ports solve a number of problems that are implicit in the use of
streams.  The following are the most obvious.
%
\begin{itemize}
\item
If several clients are to access the same object, their streams of
messages have to be merged into a single input stream.  With ports, no
merger has to be created.  Any client can send a message on the same
port.
\item
If objects are to be embedded in other data structures, creating,
e.g., an array of objects, streams have to be put in these structures.
Such structures cannot be shared, since several messages cannot be
sent on the same stream by different clients.  However, several
messages can be sent on the same port, which means that ports can be
embedded.
\item
With naive binary merging of streams, message sending delay is
variable.  With ports, message sending delay is constant.
\item
Objects based on streams require that the streams are closed when the 
clients stop using them.  This is similar to decrementing a reference 
counter, and has similar problems, besides being unnecessarily explicit 
and low-level.  A port is automatically closed when there are no more 
potential senders, thus notifying the object consuming messages.
\end{itemize}

\section{Functions and Relations}

Functions and relations are simple but powerful mathematical concepts.
Many programming languages have been designed so that one of the
available interpretations of a procedure definition should be a
function or a relation.  AKL has well-defined subsets that enjoy such
interpretations, and provide the corresponding programming paradigms.

\subsection{Functions}

The functional style of programming is characterised by the
determinate flow of control and by the non-cyclic flow of data.  There
is no don't care or don't know nondeterminism: a single result is
computed; and agents do not communicate bi-directionally; an agent
takes input from one agent and produces output to another agent.  The
latter point is weakened somewhat if the language has a non-strict
semantics, in which case ``tail-biting'' techniques are possible.

Many of the AKL definitions are indeed written in the functional
style.  For example, the ``{\prog append}'', ``{\prog squares}'' and
``{\prog fruits}'' definitions in the preceding sections are
essentially functional, although the latter two were introduced as
components in a process-oriented setting.

The basic relation between functional programs and AKL definitions is
illustrated by an example, written in the non-strict, purely
functional language Haskell.  (The appropriate type declarations are
supplied with the functional program for clarity.)
%
\begin{program}
data (BinTree a) => (Leaf a) | (Node (BinTree a) (BinTree a)) \nl
flatten :: (BinTree a) -> [a] \nl
flatten (Leaf x) l = x:l \\
flatten (Node x y) l = flatten x (flatten y l)
\end{program}%
%
In AKL, a corresponding program is phrased as follows.
%
\begin{program}
flatten(leaf(X), L, R) :-- \\
\>\cond\>	R = [X|L].  \\
flatten(node(X, Y), L, R) :-- \\
\>\cond\>	flatten(Y, L, L$_1$), \\
\>\>	flatten(X, L$_1$, R).
\end{program}%
%
The main differences are that an explicit argument has to be supplied
for the output of the ``function'', and that nested function
applications are un-nested, making the output of one the input of
another.

AKL is not a higher-order language, and does not provide ``definition
variables'', but does provide the same functionality (modulo currying)
in a simple manner.  The technique has been known in logic programming
for a long time [Warren 1981].  A term representation is chosen for
each definition in a program, and an agent apply is defined, which
given such a term applies it to arguments and executes the
corresponding definition.

One possible scheme for AKL is as follows.  Let a term {\prog p($n$,
t$_1$, \dots, t$_m$)} represent a definition {\prog p/$(n-m)$}, which
when applied to $n-m$ arguments {\prog t$_{m+1}$, \dots, t$_n$} calls
{\prog p/$n$} with {\prog p(t$_1$,
\dots, t$_n$)}.

To give an example relating to the above programs, the term {\prog
flatten(3) corresponds} to the function {\prog flatten,} and the term
{\prog flatten(3, Tree)} to the function {\prog (flatten tree)} (where
{\prog Tree} and {\prog tree} are equivalent trees).  A corresponding
agent
%
\begin{program}
apply(flatten(3), [X,Y,Z]) :-- \\
\>\cond\>	flatten(X, Y, Z).  \\
apply(flatten(3,X), [Y,Z]) :-- \\
\>\cond\>	flatten(X, Y, Z).  \\
apply(flatten(3,X,Y), [Z]) :-- \\
\>\cond\>	flatten(X, Y, Z).  \\
apply(flatten(3,X,Y,Z), []) :-- \\
\>\cond\>	flatten(X, Y, Z).
\end{program}%
%
is also defined.  In practice, it is convenient to regard apply as
being defined implicitly for all definitions in a program, which is
also easily achieved in an implementation.  This functionality may now
be used as in functional programs as follows.  We define an agent
{\prog map/3}, which maps a list to another list.
%
\begin{program}
map(P, [], Ys) :-- \\
\>\cond\>	Ys = [].  \\
map(P, [X|Xs], Ys0) :-- \\
\>\cond\>	Ys0 = [Y|Ys], \\
\>\>	apply(P, [X, Y]), \\
\>\>	map(P, Xs, Ys).
\end{program}%
%
and may then call it with, e.g., {\prog map(append(3,[a]), [[b],[c]],
Ys)} and get the result {\prog Ys~=~[[a,b],[a,c]]}.

Although by no means necessary, expressions corresponding to lambda 
expressions can also be introduced.  Let an expression
%
\begin{progex}
(X$_1$, \dots, X$_k$)$\setminus$A
\end{progex}%
%
where {\prog A} is an AKL agent with free variables {\prog Y$_1$}, \dots,
{\prog Y$_m$}, stand for a term
%
\begin{progex}
p($(m+k)$, Y$_1$, \dots, Y$_m$)
\end{progex}%
%
where {\prog p/$(m+k)$} is a new agent defined as
%
\begin{progex}
p(Y$_1$, \dots, Y$_m$, X$_1$, \dots, X$_k$) := A.
\end{progex}%
%
We may now write, e.g., {\prog map((X,Y)$\setminus$append(X, Z, Y),
[[b],[c]], Ys)} and get the result {\prog Ys~=~[[b|Z],[c|Z]]}.
Finally, the syntactic gap to the functional notation can be closed
even further by introducing the syntax
%
\begin{progex}
P(X$_1$, \dots, X$_k$)
\end{progex}%
%
standing for
%
\begin{progex}
apply(P, [X$_1$, \dots, X$_k$])
\end{progex}%
%
Obviously, the terms corresponding to functional closures may be given
more efficient representations in an implementation.

\subsection{Relations}

The relational paradigm is known from logic programming as well as
from from database query languages.  Most prominent of logic
programming languages is Prolog, which is entirely based on the
relational paradigm.  A large number of powerful programming
techniques have been developed.  Prolog and its derivatives are used
for data and knowledge base applications, constraint satisfaction, and
general symbolic processing.  AKL supports Prolog-style programming.

Characteristic of the relational paradigm is the idea that programs
interpreted as defining relations should be capable of answering
queries involving these relations.  Thus, if a parent relation is
defined, the program should be able to produce all parents for given
children and all children for given parents, enumerate all parents and
corresponding children, and verify given parents and children.  The
following definition clearly satisfies this condition.
%
\begin{program}
parent(sverker, adam).  \\
parent(kia, adam).  \\
parent(sverker, axel).  \\
parent(kia, axel).  \\
parent(jan_christer, sverker).  \\
parent(hillevi, sverker).
\end{program}%
%
Maybe less intuitive, but just as appealing, is the following: a
simple parser of a fragment of the English language.  The creation of
a parse-tree is omitted.
%
\begin{program}
s(S$_0$, S) := np(S$_0$, S$_1$), vp(S$_1$, S).  \nl
np(S$_0$, S) := article(S$_0$, S$_1$), noun(S$_1$, S).  \nl
article([a|S], S).  \\
article([the|S], S).  \nl
noun([dog|S], S).  \\
noun([cat|S], S).  \nl
vp(S$_0$, S) := intransitive_verb(S$_0$, S).  \nl
intransitive_verb([sleeps|S], S).  \\
intransitive_verb([eats|S], S).
\end{program}%
%
The two arguments of each atom represent a string of tokens to be
parsed as the difference between the first and the second
argument.  The following is a sample execution.
%
\begin{progex}
s([a,\,dog,\,sleeps], S) \\
np([a,\,dog,\,sleeps], S$_2$), vp(S$_2$, S) \\
article([a,\,dog,\,sleeps], S$_1$), noun(S$_1$, S$_2$), vp(S$_2$, S) \\
noun([dog,\,sleeps], S$_2$), vp(S$_2$, S) \\
vp([sleeps], S) \\
intransitive_verb([sleeps], S) \\
S = []
\end{progex}%
%
The relation defined by {\prog s} is
%
\begin{progex}
s([a,\,dog,\,sleeps|S], S) \\
s([a,\,dog,\,eats|S], S) \\
s([a,\,cat,\,sleeps|S], S) \\
s([a,\,cat,\,eats|S], S) \\
s([the,\,dog,\,sleeps|S], S) \\
s([the,\,dog,\,eats|S], S) \\
s([the,\,cat,\,sleeps|S], S) \\
s([the,\,cat,\,eats|S], S)
\end{progex}%
%
for all {\prog S}, and will be generated as alternative results from
%
\begin{progex}
s(S$_0$, S)
\end{progex}%
%
The idea of a pair of arguments representing the difference between
lists is important enough to warrant syntactic support in Prolog, the
DCG syntax, which allows the above definitions to be rendered as
follows.
%
\begin{program}
s --> np, vp.  \nl
np --> article, noun.  \nl
article --> [a].  \\
article --> [the].
\end{program}%
%
and so on.  The example is naive, since real examples would be
unwieldy, but the state of the art is well advanced, and the
literature on unification grammars based on the above simple idea is
rich and flourishing.

\section{Constraint Programming}

Many interesting problems in computer science and neighbouring areas
can be formulated as constraint satisfaction problems (CSPs).  To
these belong, for example, Boolean satisfiability, graph colouring,
and a number of logical puzzles (a couple of which will be used as
examples).  Other, more application oriented, problems can usually be
mapped to a standard problem, e.g., register allocation to graph
colouring.  Often, these problems are NP-complete; any known general
algorithm will require exponential time in the worst case.  Our task
is to write programs that perform well in as many cases as possible.

A CSP can be defined in the following way.  A {\em (finite) constraint
satisfaction problem} is given by a sequence of variables {\prog
X$_1$}, \dots, {\prog X$_n$}; a corresponding sequence of (finite)
domains of values {\prog D$_1$}, \dots, {\prog D$_n$}; and a set of
constraints {\prog c(X$_{i_1}$, \dots, X$_{i_k}$)}.  A {\em solution}
is an assignment of values to the variables, from their corresponding
domains, which satisfies all the constraints.

For our purposes, a constraint can be regarded as a logical formula,
where satisfaction corresponds to the usual logical notion, but
formalism will not be pressed here.  Instead, AKL programs are used to
describe CSPs, and their intuitive logical reading provides us with
the corresponding constraints.  Each agent is regarded as a
(user-defined) constraint, and will be referred to as such.  The
agents are typically don't know nondeterministic, and the assignments
for which the composition of these agents does not fail are the
solutions of the CSP.

The example to be used in this section is the $n$-queens problem: how
to place $n$ queens on an $n$ by $n$ chess board in such a way that no
queen threatens another.  The problem is very well known, and no new
algorithm will be presented.  The novelty, compared to solutions in
conventional languages, lies in the way the algorithm is expressed.
The technique used is due to Saraswat \cite{sar93}.

Each square of the board is a variable {\prog V}, which takes the
value {\prog 0} (meaning that there is no queen on the square) or
{\prog 1} (meaning that there is a queen on the square).

The basic constraint is that there may be at most one queen in each
row, column, and diagonal.  Given that $n$ queens are to be placed on
an $n$ by $n$ board, a derived constraint, which we will use, is that
there must be exactly one queen in each row and column.  Note that the
exactly-one constraint can be decomposed into an at-least-one and an
at-most-one constraint.  We now proceed to define these constraints in
terms of smaller components.  The problem is not only to express the
constraints, which is easy, but to express them in such a way that an
appropriate level of propagation will occur, which will reduce the
search space dramatically.

The at-most-one constraint can be expressed in terms of the following
agent.
%
\begin{program}
xcell(1, N, N).  \\
xcell(0, _, _).
\end{program}%
%
Note that this agent is determinate if the first argument is known, or
if the last two arguments are known and different.  For a sequence of
squares {\prog V$_1$} to {\prog V$_k$,} we can now express that at most one
of these squares is {\prog 1} using the {\prog xcell} agent as follows.
%
\begin{progex}
xcell(V$_1$, N, 1), \\
xcell(V$_2$, N, 2), \\
\dots, \\
xcell(V$_k$, N, k)
\end{progex}%
%
If more than one {\prog V$_i$} is {\prog 1}, the variable {\prog N}
will be bound to two different numbers, and the constraint will fail.
Let us call this constraint {\prog at_most_one(V$_1$, \dots, V$_k$)}, thus
avoiding the overhead of having to write a program to create it.

An {\prog at_most_one constraint} will clearly only have solutions
where at most one square is given the value {\prog 1}, but note also
the following propagation effects.  If one of the {\prog V$_i$} is
given the value {\prog 1}, its associated {\prog xcell} agent becomes
determinate, and can therefore be reduced.  When it is reduced, {\prog
N} is given the value $i$, and the other xcell agents become
determinate, and can be reduced, giving their variables the value
{\prog 0}.

The at-least-one constraint can be expressed in terms of the following
agent.
%
\begin{program}
ycell(1, _, _).  \\
ycell(0, S, S).
\end{program}%
%
Note that this agent too is determinate if the first argument is
known, or if the other two arguments are known and different.  For a
sequence of squares {\prog V$_1$} to {\prog V$_k$}, we can express that at
least one of these squares is {\prog 1} using the {\prog ycell} agent
as follows.
%
\begin{progex}
S$_0$ = begin, \\
ycell(V$_1$, S$_0$, S$_1$), \\
ycell(V$_2$, S$_1$, S$_2$), \\
\dots, \\
ycell(V$_k$, S$_{k-1}$, S$_k$), \\
S$_k$ = end
\end{progex}%
%
If all the squares are {\prog 0}, a chain of equality constraints,
{\prog S$_0$~=~S$_1$}, {\prog S$_1$~=~S$_2$,} \dots, will connect
``{\prog begin}'' with ``{\prog end}'' by equality constraints, and
the constraint will fail.  This constraint we call {\prog
at_least_one(V$_1$, \dots, V$_k$)}.

Again note the propagation effects.  If a variable is given the value
{\prog 0}, then its associated {\prog ycell} agent becomes
determinate.  When it is reduced, its second and third arguments are
unified.  If all variables but one are {\prog 0}, the second argument
of the remaining {\prog ycell} agent will be ``{\prog begin}'' and its
third argument will be ``{\prog end}'', and it will therefore be
determinate.  When it is reduced, its first argument will be given the
value {\prog 1}.

Thus, not only will these constraints avoid the undesirable cases, but
they will also detect cases where information can be propagated.  When
no agent is determinate, and therefore no information can be
propagated, alternative assignments for variables will be explored by
trying alternatives for the {\prog xcell} and {\prog ycell} agents.

A program solving the $n$-queens problem can now be expressed as
follows.
%
\begin{itemize}
\item
For each column, row, and diagonal, consisting of a sequence of
variables {\prog V$_1$}, \dots, {\prog V$_k$}, the constraint {\prog
at_most_one(V$_1$, \dots, V$_k$)} has to be satisfied.
\item
For each column and row, consisting of a sequence of variables {\prog
V$_1$}, \dots, {\prog V$_n$}, the {\prog constraint at_least_one(V$_1$, \dots,
V$_n$)} has to be satisfied.
\item
The composition of these constraints is the program.
\end{itemize}
%
Note that when information is propagated, this will affect other
agents, making them determinate.  This will often lead to new
propagation.  One such case is illustrated below.
%
\begin{center}
\begin{tabular}{|*{4}{p{0.6cm}|}}
\hline
1 & 0 & 0 & 0 \\
\hline
0 & 0 & V$_{23}$ & V$_{24}$ \\
\hline
0 & V$_{32}$ & 0 & V$_{34}$ \\
\hline
0 & V$_{42}$ & V$_{43}$ & 0 \\
\hline
\end{tabular}
\end{center}
%
The above grid represents the board, and in each square is written the
variable representing it, or its value if it has one.  We will now
trace the steps leading to the above state.  Initially, all variables
are unconstrained, and all the constraints have been created.  Let us
now assume that the topmost leftmost variable ({\prog V$_{11}$}) is
given the value {\prog 1}.  It appears in the row {\prog V$_{11}$} to
{\prog V$_{14}$}, in the column {\prog V$_{11}$} to {\prog V$_{41}$},
and in the diagonal {\prog V$_{11}$} to {\prog V$_{44}$}.  Each of
these is governed by an {\prog at_most_one} constraint.  By giving one
variable the value {\prog 1}, the others will be assigned the value
{\prog 0} by propagation.

A second case of propagation is the following, where {\prog V$_{12}$}
and {\prog V$_{24}$} are assumed to contain queens, and propagation of
the above kind has taken place.
%
\begin{center}
\begin{tabular}{|*{4}{p{0.6cm}|}}
\hline
0 & 1 & 0 & 0 \\
\hline
0 & 0 & 0 & 1 \\
\hline
V$_{31}$ & 0 & 0 & 0 \\
\hline
V$_{41}$ & 0 & V$_{43}$ & 0 \\
\hline
\end{tabular}
\end{center}
%
Here we examine the propagation that this state will lead to.  Notice
that in row 3, all variables but {\prog V31} have been given the value
{\prog 0}.  This triggers the {\prog at_least_one} constraint
governing this row, giving the last variable the value {\prog 1},
which in turn gives the variables in the same row, column, or diagonal
(only V$_{41}$) the value {\prog 0}.  Finally, {\prog V$_{43}$} is
given the value {\prog 1} by reasoning as above.

The above program is reasonably good.  The programs usually written
for constraint logic programming languages with finite domain
constraints do not exploit the fact that both rows and columns should
contain exactly one queen (e.g., \cite{hen89}).  A very good
solution can be obtained if the xcell and ycell agents are ordered so
that those governing variables closer to the centre of the board come
before those governing variables further out.  If at some step
alternatives have to be tried for an agent, values will be guessed for
variables at the centre first.  This happens to be a good heuristic
for the n-queens problem, even better than the so called first-fail
principle, which is usually employed.

\section{Integration}

So far, the different paradigms have been presented one at a time, and
it is quite possibly by no means apparent in what relation they stand
to each other.  In particular the relational and the constraint
satisfaction paradigms have no apparent connection to the process
paradigm.  Here, this apparent dichotomy will be bridged, by showing
how a process-oriented application based on the solver for the
$n$-queens problem could be structured.

The basic techniques for interaction with the environment (e.g., files
and the user) are shown first, and then a program structure is
introduced which is somewhat inspired by the Smalltalk
Model-View-Controller paradigm.

\subsection{Interoperability}

The idea underlying interoperability is that an AKL agent sees itself
as living in a world of AKL agents.  The user, files, other programs,
all are viewed as AKL agents.  If they have a state, e.g., file
contents, they are closer to objects, such as those discussed
above.  It is up to the AKL implementation to provide this view, which
is inherited from the concurrent logic programming languages.

A program takes as parameter a port to the ``operating system'' (OS)
agent, which provides further access to the functionality and
resources it controls.  An interface to foreign procedures adds glue
code that provides the necessary synchronisation, and views of mutable
data structures as ports to agents.  The examples use imaginary,
although realistic, primitives, as in the following.
%
\begin{program}
main(P) := \\
\>\>	send(create_window(W, [xwidth=100, xheight=100]), P), \\
\>\>	send(draw_text(10, 10, 'Hello, world!'), W).
\end{program}%
%
Here it is assumed that the agent main is supplied with the
``operating system'' port P when called.  The OS provides window
creation, an operation that returns a port to the window agent, which
provides text drawing, and so on.

For some kinds of interoperability, a consistent view of don't know
nondeterminism can be implemented.  For example, a sub-program without
internal state, such as a numerical library written in C, does not
mind if its agents are copied during the course of a computation.  For
particular purposes, it is even possible to copy windows and similar
``internal'' objects.  But the real world does not support don't know
nondeterminism.  It would hardly be possible to copy an agent that
models the actual physical file-system; nor would it be possible to
copy an agent that models communication with another computer.

The only solution is to regard this kind of incompleteness as
acceptable, and either let attempts to copy such unwieldy agents
induce a run-time error, or give statements a ``type'' which is
checked at compile-time, and which shows whether a statement can
possibly employ don't know nondeterminism.

\subsection{Encapsulation}

To avoid unwanted interaction between don't know nondeterministic and
pro\-cess-oriented parts of a program, the nondeterministic part can be
encapsulated in a statement that hides nondeterminism.  Nondeterminism
is encapsulated in the guard of a conditional or committed choice and
in bagof.  When encapsulated in a guard, a nondeterministic
computation will eventually be pruned.  In a conditional choice, the
first solution is chosen.  In a committed choice, any solution may be
chosen.  When encapsulated in bagof, all solutions will be collected
in a list.

More flexible forms of encapsulation can be based on the notion of
engines.  An engine is conceptually an AKL interpreter.  It is situated
in a server process.  A client may ask the engine to execute programs,
and, depending on the form of engine, it may interact with the engine
in almost any way conceivable, inspecting and controlling the
resulting computation.  A full treatment of engines for AKL is future
work.

\subsection{Model-View-Controller}

The Model-View-Controller (MVC) paradigm for assigning different
responsibilities to the components of an object-oriented program is
easily realised in AKL, as in any object- or process-oriented
language.  In AKL it also localises and encapsulates don't know
nondeterminism in the relevant part of the program, which is usually
the model.
                                     
% \verb|Figure 3.4  The Model-View-Controller paradigm|

In the next section, MVC will be applied to the structuring of an
$n$-queens application, using imaginary OS primitives.

\subsection{An N-Queens Application}

Assume the existence of a don't know nondeterministic $n$-queens agent
%
\begin{program}
n_queens(N, Q) := \dots.
\end{program}%
%
which returns different assignments {\prog Q} to the squares of an
{\prog N} by {\prog N} chess board.  It is easily defined by adding
code for creation of constraints for different length sequences, and
code for creating sequences of variables corresponding to rows,
columns, and diagonals on the chess board.  No space will be wasted on
this trivial task here.  We proceed to the MVC structure with which to
support the application.
%
\begin{program}
main(P) := \\
\>\>	initialise(P, W, E), \\
\>\>	view(V, W), \\
\>\>	controller(E, M, S, V), \\
\>\>	model(M, S).
\end{program}%
%
The initialise agent creates a window accepting requests on stream
{\prog W} and delivering events on stream {\prog E}.  The view agent
presents whatever it is told to by the controller on the window using
stream {\prog W}.  The model delivers solutions on the stream {\prog
S} to the n-queens problems submitted on stream {\prog M}.  The
controller is driven by the events coming in on {\prog E}.  It submits
problems to the model on stream {\prog M} and receives solutions on
stream {\prog S}.  It then sends solutions to the view agent on {\prog
V} for displaying.

Let us here ignore the implementation of the initialise, view, and
controller agents.  The interesting part is how the don't know
nondeterminism is encapsulated in the model agent.  We assume that we
are satisfied with being able to get either one or all solutions from
the particular instance of the n-queens problem, or getting the reply
that there are no solutions (for {\prog N~=~2} or {\prog N~=~3)}.
%
\begin{program}
model([], S) :-- \\
\>\cond\>	S = [].  \\
model([all(N)|M], S) :-- \\
\>\cond\>	bagof(Q, n_queens(N, Q), Sols), \\
\>\>	S = [all(Sols)|S$_1$], \\
\>\>	model(M, S$_1$).  \\
model([one(N)|M], S) :-- \\
\>\cond\>	( Q :\>\>n_queens(N, Q) \\
\>\>		\>\cond\>S = [one(Q)|S$_1$] \\
\>\>	; 	\>\>S = [none|S$_1$] ), \\
\>\>	model(M, S$_1$).
\end{program}%
%
As described above, don't know nondeterminism within bagof and choice
statements is not propagated further.  The MVC part of the program can
be kept entirely free of nondeterminism.

\section{Current and Future Work}

Current and planned topics at SICS include efficient sequential and
parallel implementations parametrised with user-definable constraint
systems (in C), implementations of various constraint systems,
extensions of the basic framework, such as engines for meta-level
programming, program analysis and program transformation,
inter-operability with conventional languages and operating systems,
and investigation of formal properties.

An experimental AKL programming system is available from SICS for
research and educational purposes.

\subsection*{Acknowledgements}

The authors wish to thank the other members of the Concurrent
Constraint Programming group at SICS for their contributions to this
work.  Discussions with Vijay Saraswat and David~H.~D. Warren during
the design phase were very valuable.

\bibliographystyle{plain}
\bibliography{abib}

\end{document}
